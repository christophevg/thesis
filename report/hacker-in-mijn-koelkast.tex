\documentclass[DIV=calc,paper=a4,fontsize=11pt,twocolumn,draft]{scrartcl}

\usepackage{lipsum}
\usepackage[dutch]{babel}
\usepackage[protrusion=true,expansion=true]{microtype}
\usepackage{amsmath,amsfonts,amsthm}
\usepackage[svgnames]{xcolor}
\usepackage[hang, small,labelfont=bf,up,textfont=it,up]{caption}
\usepackage{booktabs}
\usepackage{fix-cm}
\usepackage{sectsty}
\allsectionsfont{\usefont{OT1}{phv}{b}{n}}

\usepackage{fancyhdr}
\pagestyle{fancy}
\usepackage{lastpage}

\lhead{}
\chead{}
\rhead{}
\lfoot{\footnotesize Deze ochtend vond ik een hacker in mijn koelkast - Christophe Van Ginneken}
\cfoot{}
\rfoot{\footnotesize \thepage}

\renewcommand{\headrulewidth}{0.0pt}
\renewcommand{\footrulewidth}{0.4pt}

\usepackage{lettrine}
\newcommand{\initial}[1]{
\lettrine[lines=4,lhang=0.3,nindent=0em]{
\color{DarkGoldenrod}
{\textsf{#1}}}{}}

\usepackage{titling}

\newcommand{\HorRule}{\color{DarkGoldenrod} \rule{\linewidth}{1pt}}

\pretitle{\vspace{-30pt} \begin{flushleft} \HorRule \fontsize{50}{50}
\usefont{OT1}{phv}{b}{n} \color{DarkRed} \selectfont}

\title{Deze ochtend vond ik een hacker in mijn koelkast}

\posttitle{\par\end{flushleft}\vskip 0.5em}

\preauthor{\begin{flushleft}\large \lineskip 0.5em \usefont{OT1}{phv}{b}{sl}
\color{DarkRed}}
\author{Christophe Van Ginneken, }
\postauthor{\footnotesize \usefont{OT1}{phv}{m}{sl} \color{Black}
Student MSc Computerwetenschapp, KULeuven
\par\end{flushleft}\HorRule}

\date{\today}

\setlength{\columnsep}{1cm}

\usepackage{lineno}
\linenumbers

\begin{document}
\thispagestyle{fancy}

\twocolumn[
\begin{@twocolumnfalse}
\maketitle
\begin{abstract}

De opkomst van het ``internet van de dingen'' is niet meer te stoppen en weldra
zal elk ding op aarde voorzien zijn van een eigen plek op het internet. Dit is
leuk, want zo zullen we van op het werk kunnen nakijken hoeveel melk er nog in
onze koelkast staat of kunnen we de verwarming van de badkamer alvast een
graadje hoger zetten terwijl we nog in de file zitten op een druilige
vrijdagavond. Maar terwijl wij kunnen genieten van al deze luxe, staat de
voordeur van ons geautomatiseerd huis ook open voor anderen, met minder
aangename bedoelingen en een massa aan mogelijkheden om al onze online
\emph{dingen} aan te vallen. Zullen we ons kunnen verdedigen, of is de strijd
bij voorbaat al verloren?

\end{abstract}
\end{@twocolumnfalse}
]

\initial{H}et lijkt een deel van een dialoog uit een slechte science fiction
film. Als we echter de evolutie van technologie van naderbij bekijken, zien we
dat de realiteit misschien sneller dan verwacht de fictie zal inhalen.

Nu het internet er voor gezorgd heeft dat elke computer - en ondertussen ook
bijna elke telefoon - ter wereld online is, voelen we reeds de hete adem van de
volgende revolutie in onze nek. Opnieuw wordt de schaal weer kleiner en wil men
kost wat kost elk ding ter wereld aansluiten op internet.

Elk ding mogen we hier eigenlijk zelfs letterlijk nemen. Onder de noemer van
het ``internet van de dingen'' wordt al enkele jaren getracht om elk toestel in
ons huis te voorzien van een aansluiting naar het internet. De eerste stap was
de introductie van digitale televisie. Deze ``digiboxen'' zijn in essentie
kleine computers verbonden met de televisiedistributeur via het internet. Via
interactieve programmagidsen en spelletjes, verrijken ze onze
televisie-ervaring met een ``druk op de rode knop''.

De makers van televisietoestellen konden niet achter blijven en al snel zat er
ook in onze televisietoestellen een netwerkaansluiting en konden we surfen
zonder ook maar van scherm te veranderen. Via diezelfde televisietoestellen
vernemen we vandaag dat we dankzij onze electriciteitsleverancier nu ook het
energieverbruik van elk toestel in ons huis kunnen controleren en indien nodig
onze hond even een lesje leren.

Het succes van deze mogelijkheden werkt aanstekelijk en overal hoor je verhalen
over hoe fijn het zou zijn als we ons hele leven zouden kunnen besturen via het
internet. Waarom zouden we immers onze koelkast niet op het internet aansluiten
en zo in staat stellen om ons te laten weten dat we nog melk moeten halen of
dat die schimmelkaas nu echt wel ... aan vervanging toe is.

\subsubsection*{De realiteit achtervolgt de fictie}
\vspace{-3mm}

We schrijven midden augustus 2013. Ergens in de Verenigde Staten van Amerika,
leggen twee jonge ouders hun kind te slapen, onder het alziende oog van hun
nieuwe draadloze, met het internet verbonden, babyfoon. Wanneer zij enige tijd
later de kamer van het kind opnieuw betreden, horen ze een onbekende stem
obscene woorden ten berde brengen langs deze babyfoon, tot groot jolijt van de
kleine spruit.

Ondertussen wordt in Europa duchtig verder gesleuteld aan de draadloze
pacemaker. Een wonderbaarlijk stukje technologie dat artsen toegang geeft tot
het hart van hun pati\"ent, waar deze zich ook bevindt. Het lijkt wel een scene
uit een fictie-serie waarin een hacker zich toegang verschaft tot zo'n
pacemaker en zo de drager ervan vermoordt. Fictie? Dick Cheney denkt het in
ieder geval niet. In oktober 2013 heeft hij immers het draadloze aspect van
zijn pacemaker laten verwijderen, om een mogelijke terroristische aanval te
vermijden.

Toen we in 1995 genoten van de eerste acteerprestaties van Angelina Jolie in de
cultfilm ``Hackers'', leek het kunnen besturen van verkeerslichten een prachtig
staaltje science fiction. Bijna 20 jaar later, zijn verkeerslichten en
draadloze sensornetwerken in wetenschappelijke publicaties alvast dikke
vrienden en is men klaar om onder het mom van zgn. ``slimme steden'', elk van
deze lichten een autonoom leven te bieden. De grote omarming door het
``internet van de dingen'' is nu reeds alomtegenwoordig.

\subsubsection*{Een wereld vol microcontrollers en sensoren}
\vspace{-3mm}

TODO

\subsubsection*{De hacker in de koelkast}
\vspace{-3mm}

TODO


\lipsum[1-10]

\end{document}