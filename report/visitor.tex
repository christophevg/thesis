%!TEX root=masterproef.tex

\chapter{Het \emph{visitor} patroon}
\label{appendix:visitor}

Het \emph{visitor} patroon vormt de basis voor transformaties in de generator.
In deze bijlage belichten we dit patroon kort, alsook de implementatie in
Python, zoals deze gerealiseerd werd in het kader van deze masterproef.

\section{Het patroon}
\label{section:devel-visitor-pattern}

De kern van de transformaties is het \emph{visitor} patroon zoals beschreven in
het de facto standaardwerk ``Design Patterns: Elements of reusable
object-oriented software'' \citep{gamma1994design}. Figuur \ref{fig:visitor}
illustreert het patroon in UML vorm.

\begin{figure}[ht]
  \centering
  \includegraphics[width=0.9\linewidth]{resources/visitor.pdf}
  \caption[Het \emph{visitor} patroon in UML]{Het \emph{visitor} patroon in UML (Bron: \citep{wikipedia:visitor})}
  \label{fig:visitor}
\end{figure}

Het principe tracht een ontkoppeling te maken van een taxonomie van klassen en
de manier waarop deze klassen overlopen kunnen worden. Het patroon vraagt
daarom het bestaan van twee basis-entiteiten: een te bezoeken taxonomie met een
gemeenschappelijke basisklasse en een declaratie van een bezoekende klasse.

De gemeenschappelijke basisklasse declareert typisch een methode \ttt{accept}
die een instantie van de bezoekende klasse als parameter accepteert. De
bedoeling is dat de implementerende taxonomie vervolgens voor elke klasse
specifiek een implementatie maakt van deze methode. Deze implementatie zal
ofwel de bezoekende klasse doorspelen aan hi\"erarchisch onderliggende klassen,
ofwel een \ttt{visit} methode op de bezoekende klasse oproepen met zichzelf als
argument.

Op een implementatie van een bezoekende klasse kunnen we nu verschillende
overladen (\emph{overloaded}) methoden defini\"eren voor elk van de klassen in
de taxonomie. Op deze manier kan men verschillende bezoekende klassen
realiseren zonder aanpassing aan de bezochte klassen en moet de bezoekende
klasse ook geen kennis hebben van de structuur van de taxonomie.

\section{In Python}

Python beschikt echter niet over het concept van overladen functies. Dit gebrek
kan opgevangen worden door het type van de klasse mee op te nemen in de
\ttt{visit}-functie. Dit eist natuurlijk dat bij de oproep van de
\ttt{visit\_<class>} functie dit type mee in beschouwing wordt genomen. Zo
vervalt de mogelijkheid om met \'e\'en implementatie van de functie een
onderliggend deel van de taxonomie te voorzien van een standaard implementatie.
Figuur \ref{fig:py-visitor} geeft de opbouw van een mogelijke implementatie in
Python weer.

\begin{figure}[ht]
  \centering
  \includegraphics[width=0.9\linewidth]{resources/py-visitor.pdf}
  \caption{Een \emph{visitor} implementatie in Python}
  \label{fig:py-visitor}
\end{figure}

Dankzij het dynamische karakter van Python is het mogelijk om veel van deze
problematiek te verbergen en te automatiseren. Door gebruik te maken van
\emph{decorators} is het mogelijk om Python functies en klassen dynamisch aan
te passen voor ze gebruikt worden. Door middel van dit principe is het mogelijk
om een klasse te declareren die de klassen, gedefini\"eerd in een bepaalde
module, kan bezoeken. Van deze klasse kan vervolgens een implementatie gemaakt
worden die de structuur van de taxonomie implementeert, door op dezelfde manier
als in het klassieke patroon een \ttt{accept}-functie op te roepen.

Deze \ttt{accept}-functie is ge\"implementeerd op een basisklasse voor de
klassen in de taxonomie en gebruikt de introspectieve capaciteiten van Python
om dynamisch een functieoproep te construeren naar een
\ttt{handle\_<class>}-functie. Deze functies kunnen ge\"implementeerd worden op
een klasse die de eigenlijke functionaliteit van de \emph{visitor}
implementeert en daarvoor overerft van de eerste concrete implementatie van de
\emph{visitor}.

Deze drieledige structuur laat toe om functioneel-gerichte bezoekende
implementaties te maken. Louter de echte functionaliteit wordt opgenomen in een
functie met signatuur \ttt{handle\_<class>}. Deze simpele uitwendige interface
is ook mede de reden dat doorheen de hele implementatie van de generator vele
bezoekende klassen gebruikt worden.

Een laatste bijkomende uitbreiding die werd toegevoegd is de opsplitsing van de
\ttt{handle\_<ttype>} functie in een \ttt{before\_visit\_<class>} en
\ttt{after\_visit\_<class>}. Hierdoor kan de manier waarop doorheen de
hi\"erarchische boomstructuur wordt gelopen bepaald worden: ofwel eerst in de
breedte (\emph{Depth-First}) ofwel eerst in diepte (\emph{Breadth-First}). Dit
kan bv. belangrijk zijn indien eerst de onderliggende kinderen verwerkt moeten
worden, vooraleer de feitelijke transformerende functionaliteit mag toegepast
worden. De vertaling van het SM naar het CM is hiervan een voorbeeld.
