%!TEX root=masterproef.tex

\subsection{Detecteren van anomalie\"en}
\label{subsection:anomaly}

Een anomalie is een afwijking van een normaal verloop van gebeurtenissen of
iets of iemands gedrag. Een knoop uit een DSN heeft een redelijk eenvoudig en
constante levensloop. Typisch zal een knoop op regelmatige tijdstippen
\emph{wakker worden}, waarden opmeten aan de hand van zijn sensoren en deze
waarden doorsturen naar een centrale locatie. Verder zal een knoop, als
onderdeel van het netwerk, tevens zulke waarden van andere knopen doorsturen.
Dit patroonmatige gedrag kan ge\"identificeerd worden en in een model verwerkt
worden. Op basis van zulk een model kan vervolgens nagegaan worden of de acties
van een knoop op een gegeven moment in lijn zijn met het model of dat er sprake
is van een anomalie.

Zulk een afwijking van het verwachte gedrag kan wijzen op veranderingen van
buitenaf. Deze kunnen op hun beurt veroorzaakt zijn door een aanval op het
netwerk. Aangezien we niet alle communicatie van en naar knopen kunnen
onderscheppen en eventuele aanvallen kunnen detecteren, kunnen
anomaliegebaseerde dectectiemechanismen helpen om op basis van neveneffecten
toch aanvallen te detecteren, of althans toch de gevolgen ervan.

\subsubsection*{Anomali\"en, afwijkingen, aberraties}
\label{subsubsection:outlier}

\citep{zhang2010outlier} is een excellent overzicht van methoden om anomali\"en,
afwijkingen of aberraties (\emph{outliers}) te detecteren in een reeks
van metingen. Het belicht enerzijds de fundamentele technieken die ter
beschikking staan om deze aberraties op te merken maar tracht ook een
classificatie en taxonomie op te stellen hiervoor.

De auteurs stellen dat het detecteren van afwijkingen behoort tot het domein
van \emph{datamining} en dat het in die context reeds uitvoerig onderzocht is,
evenals binnen disciplines as statistiek, machinaal leren, informatie
theorie \dots. Aangezien het kunnen uitsluiten van afwijkingen de verwerking
van de overblijvende meetresultaten sterk positief be\"invloed is het in het
kader van DSN uitermate interessant.

Toch kan eerder onderzoek opnieuw niet eenvoudig toegepast worden in het kader
van DSN. De beperkte middelen van de sensorknoop schrappen al veel van de
klassieke oplossingen die bv. gecentraliseerd werken. De overdaad aan
communicatie die nodig is om voldoende gegevens te centraliseren voor
verwerking is niet realistisch in het kader van een DSN. Ook zijn veel van de
algoritmen typisch niet ontwikkeld met de beperkte rekenmogelijkheden van
sensorknopen. De conclusie is dat er een balans moet gevonden worden tussen de
mogelijkheden van datamining algoritmen voor het detecteren van afwijkingen en
de verhouding van hun noden ten opzichte van de middelen van de knopen.

\subsubsection*{Neurale netwerken}
\label{subsubsection:neuralnetworks}

Wanneer men denkt aan het vastleggen van een patroon en het controleren of een
bepaalde situatie voldoet aan dat patroon wordt in informatiecakringen ook snel
verwezen naar neurale netwerken. Neurale netwerken kunnen immers
\emph{getraind} worden door middel van een aantal goede (en slechte)
voorbeelden, waarna nieuwe voorbeelden kunnen gecatalogeerd worden als ook goed
of slecht. De complexiteit van het bepalen van deze beslissing is typisch
redelijk eenvoudig en lijkt zich daarom uitermate goed te lenen voor het
detecteren van anomalie\"en door sensorknopen.

In \citep{ramesh2012wireless} volgen de auteurs deze denkpiste, maar stellen
tevens dat er betere methoden bestaan. Ze trachten twee specifieke aanvallen
het hoofd te bieden: DoS en passieve informatie vergaring en vergelijken
hierbij een aanpak op basis van een neuraal netwerk en hun eigen aanpak op
basis van encryptie op basis van symmetrische sleutels.

Ofschoon dat sommige van hun veronderstelling na\"ief zijn (zo baseren ze zich
op een gedeelde geheime sleutel van 8 bits), toont hun werk wel aan dat een
aanpak met neurale netwerken eenvoudig te realiseren is en een valabele piste
kan zijn om anomaliedetectie te doen.

\subsubsection*{Voorspellingen}
\label{subsubsection:predictions}

Waar neurale netwerken in staat zijn om op basis van voorbeelden een nieuwe
situatie te catalogeren, kan men aan de hand van een Markov model
voorspellingen doen over de toekomst.

Het is deze piste dat onderzocht wordt in \citep{zhijie2012intrusion}. Opnieuw
betreft het een poging om DoS aanvallen te detecteren. De bedoeling is dat
sensorknopen individueel bepalen of er een DoS aanval bezig is. Volgens de
auteurs is dit mogelijk aan de hand van een Markov model dat het netwerkverkeer
voorspelt.

Het model wordt zo geconstrueerd dat er een verband ontstaat tussen de toestand
van een knoop in relatie tot het tijdstip en de verwachtte hoeveelheid gegevens
die verstuurd zouden kunnen worden.

Het idee achter het artikel lijkt een mogelijke piste, maar omtrent veel
belangrijke details blijven echter zeer vaag, waardoor de volledige toedracht
van het algoritme niet eenduidig ingeschat kan worden. Zo wordt bv. nauwelijks
ingegaan op wat de toestand van een knoop juist bepaalt of hoe de
hoeveelheid gegevens die verstuurd kunnen worden wanneer een knoop zich in een
bepaalde toestand bevindt, bepaald wordt.
