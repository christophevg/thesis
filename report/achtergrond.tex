%!TEX root=masterproef.tex

\chapter{Achtergrond}
\label{chapter:achtergrond}

In het inleidende hoofdstuk werden DSN reeds kort voorgesteld. In dit hoofdstuk
wordt het kader geschetst waarbinnen deze thesis op zoek gaat naar antwoorden.
Enerzijds wordt in sectie \ref{section:landscape} het landschap van draadloze
sensornetwerken in kaart gebracht: wat typeert en onderscheid hen van andere
netwerken? Wie is de aanvaller en wat zijn zijn doelen? Waarom is het
vaststellen van inbraken een belangrijk onderzoeksdomein?

Om het fundamentele karakter van het probleem verder uit te werken, wordt in
sectie \ref{section:node-capture} aan de hand van een beperkt voorbeeld
aangetoond hoe eenvoudig en nagenoeg niet op te merken het geheugen van een
werkende knoop kan gelezen worden, zonder dat de werking van de knoop aangetast
wordt.

Tot slot van dit hoofdstuk wordt in sectie \ref{section:related} ingegaan op
een groot aanbod aan gerelateerd onderzoek. Een belangrijke ondersteunende
doelstelling van deze thesis is het in kaart brengen van de mogelijkheden en
beperkingen betreffende reeds beschreven methodes om inbraken vast te stellen.

De verschillende beschreven methodes worden gecatalogeerd en gegroepeerd op
basis van verschillende eigenschappen. Op basis van dit overzicht wordt
vervolgens een inschatting gemaakt van de mogelijke dekking die kan bereikt
worden met de bestaande oplossingen.

Bij de beschrijving van de verschillende oplossingen zal tevens kritisch
nagegaan worden in hoeverre de oplossingen in een realistische situatie
effectief bijdragen tot het detecteren van inbreuken in het netwerk.

%!TEX root=masterproef.tex
\section{Draadloze sensornetwerken}
\label{section:landscape}

\TODO

Zeer terecht stelt \cite{perrig2004security} dat de eerste zorg omtrent de
beveiliging van draadloze sensor netwerken een veilige manier om de groep van
sensoren te beheren is. De manier waarop nieuwe knopen in het netwerk worden
opgenomen en de beveiliging van de onderlinge communicatie is van primordiaal
belang. Voorkomen is beter dan genezen.

Maar we moeten ook realistisch zijn. Geen door mensenhanden gemaakt systeem is
feilloos en nagenoeg elk ge\"informatiseerd system zal met de nodige inzet en
moeite veroverd kunnen worden. Op dat ogenblik is het belangrijk dat er een
tweede verdedigingslinie is: de inbraak detectie.

\TODO

In \cite{zhang2000intrusion} wordt een zeer algemene, maar zeer correcte
definitie gegeven van inbraak detectie.

\begin{quote}
Intrusion detection therefore involves capturing audit data and reasoning
about the evidence in the data to determine whether the system is under attack.
\end{quote}


%!TEX root=masterproef.tex

\section{Knoopverovering}

Zoals reeds beschreven in de inleiding, is de fysieke toegankelijkheid van
knopen een re\"eel en groot probleem voor draadloze sensornetwerken. In de
volgende paragrafen tonen we aan hoe eenvoudig het is om een knoop te veroveren
en dat de knoop en zijn netwerk hier nagenoeg niets van kunnen merken.

\subsection{Situatie en doel}

Het hart van elke draadloze sensornetwerk knoop is de microcontroller of \mcu.
Door zijn ge\"integreerde architectuur bevat deze nagenoeg alle onderdelen die
interessant kunnen zijn en kan men zich louter hierop focussen bij een poging
om de knoop te veroveren.

In de testopstelling voor dit experiment maken we gebruik van een Atmel
ATMEGA1284p \cite{datasheet:atmega1284p}. Dit is een representatieve \mcu met
128KB programmeerbaar geheugen en 16KB werkgeheugen, die bv. gebruikt wordt in
de populaire Atmel RZRAVEN ontwikkelingskit \cite{manual:rzraven}. We kiezen
ervoor om de ATMEGA1284p volledig te ontdoen van enige context, om zo tevens de
algemeenheid van het probleem te illustreren.

De \mcu wordt voorzien van een eenvoudig programma dat een numerieke teller
verhoogt. Het doel van deze veroveringspoging is om de waarde van deze teller
te verkrijgen, zonder dat dit de werking van de \mcu verandert.

Figuur \ref{fig:node-capture-schematic} toont het schema van deze opstelling.

\begin{figure}[hb]
  \centering
  \includegraphics[width=0.7\linewidth]{resources/node-capture-schematic.pdf}
  \caption{Schema van de testopstelling voor knoopverovering.}
  \label{fig:node-capture-schematic}
\end{figure}

De \mcu is via een USART\footnote{USART staat voor \emph{Universal
synchronous/asynchronous receiver/transmitter} dat in staat voor de vertaling
van gegevens tussen een parallelle en seri\"ele voorstelling.} poort verbonden
met een MAX232 \cite{datasheet:max232} die de USART signalen omzet naar een
RS-232\footnote{RS-232 is een seri\"ele communicatiestandaard, typisch gebruikt
tussen computers en randapparatuur, maar ook kan dienen als eenvoudige data
verbinding tussen twee computers.} compatibele communicatie.

Listing \ref{lst:node-capture} toont de functionaliteit die op de \mcu
ge\"implementeerd werd: een globale variabele, \ttt{counter}, wordt eindeloos
verhoogd. Na elke verhoging wordt de waarde via de USART en de RS-232
verbinding naar een terminal verzonden. 

\inputminted[linenos,frame=lines,framesep=2mm,fontsize=\footnotesize]{c}{../src/node-capture/main.c}
\vspace{-5mm}
\captionof{listing}{Functionaliteit van de testopstelling voor knoopverovering.
  \label{lst:node-capture}}
\vspace{3mm}

Het is dus de bedoeling om de waarde van deze \ttt{counter} variabele te
bemachtigen, zonder dat de werking van het programma onderbroken wordt. Deze
variabele staat hier natuurlijk symbool voor eender welk gegeven dat in het
geheugen van de node wordt opgeslagen.

\subsection{Uitvoeren van de aanval}

De aanval bestaat er in dit geval in om de knoop te verbinden met een
hardwarematige foutopspoorder, zoals de Atmel JTAGICE mkII
\cite{manual:jtagicemkii}. Dit kan bv. gebeuren aan de hand van een JTAG
verbinding. Dit is een verbinding met tien draden, waarvan er vier aangesloten
moeten worden op de \mcu\footnote{Normaal gezien worden er 5 aangesloten. De
vijfde verbinding is de zgn. \ttt{RESET} aansturing. Aangezien we bij deze
aanval de \mcu zeker niet willen resetten, kan deze verbinding weggelaten
worden.} en twee naar een voeding en aarding geleid moeten worden. Zoals te
zien is op figuur \ref{fig:node-capture-jtag} worden de vier draden eenvoudig
op naast elkaar liggende pinnen (24 tot 27) van de \mcu aangesloten.

\begin{figure}[hb]
  \centering
  \includegraphics[width=0.7\linewidth]{resources/node-capture-jtag.pdf}
  \caption{Aansluiting van een JTAG verbinding.}
  \label{fig:node-capture-jtag}
\end{figure}

De uitvoer van de applicatie wordt weergegeven in listing
\ref{lst:node-capture-terminal} en toont een terminal applicatie die deze
uitvoer weergeeft. De applicatie werkt ononderbroken.

\begin{listing}
  \begin{minted}{console}
$ screen /dev/tty.usbserial-FTSJ84AI 9600
counter = 1
counter = 2
...
counter = 10
counter = 11
counter = 12
counter = 13
counter = 14
counter = 15
counter = 16
...
\end{minted}
  \caption{Uitvoer van de applicatie op de \mcu.}
  \label{lst:node-capture-terminal}
\end{listing}

Tijdens dat deze uitvoer gerealiseerd werd, werd er echter een aanval
uitgevoerd. Deze werd gedaan aan de hand van de standaard
foutopsporingsmogelijkheden van de \mcu. Hiervoor werd een aangepaste versie
van de standaard foutopsporingssoftware
\ttt{gdb}\footnote{http://www.gnu.org/software/gdb/} gebruikt, nl.
\ttt{avr-gdb}.

Aangezien deze niet standaard met een JTAG verbinding kan werken, werd tevens
een brug opgezet door middel van
\ttt{avarice}\footnote{http://avarice.sourceforge.net}. Listing
\ref{lst:node-capture-avarice} toont het opstarten van de brug en het
beschikbaar maken van de JTAG verbinding via een locale server verbinding.

\begin{listing}
  \begin{minted}{console}
$ avarice --mkII --capture --jtag usb:5a:cb :4242
AVaRICE version 2.13, Oct 29 2013 15:35:57

Defaulting JTAG bitrate to 250 kHz.

JTAG config starting.
...
Waiting for connection on port 4242.
Connection opened by host 127.0.0.1, port 58521.
  \end{minted}
  \caption{\ttt{avarice} brug tussen JTAG-gebaseerde foutopspoorder en \ttt{gdb}.}
  \label{lst:node-capture-avarice}
\end{listing}

Vervolgens kan de standaard foutopspoorder, \ttt{gdb}, gebruikt worden om het
geheugen van de \mcu te raadplegen en het programma verder te laten lopen.
Listing \ref{lst:node-capture-gdb} toont deze interactie.

\begin{listing}
  \begin{minted}{console}
$ avr-gdb
GNU gdb 6.8
...
(gdb) target remote localhost:4242
Remote debugging using localhost:4242
0x000001d8 in ?? ()
(gdb) dump binary memory counter12.bin 8388892 8388900
(gdb) c
Continuing.
^C
Program received signal SIGINT, Interrupt.
0x000001d2 in ?? ()
(gdb) dump binary memory counter15.bin 8388892 8388900
(gdb) c
Continuing.
  \end{minted}
  \caption{\ttt{gdb} interactie met de \mcu.}
  \label{lst:node-capture-gdb}
\end{listing}

Listing \ref{lst:node-capture-hexdump} toont de inhoud van de veroverde
gegevens. \ttt{0c} en \ttt{0f} zijn telkens de eerste byte in het geheugen van
de \ttt{counter} variabele en tonen inderdaad de waarde die de variabele had op
het moment van de opvraging.

\begin{listing}
  \begin{minted}{console}
$ hexdump counter12.bin 
0000000 0c 00 00 00 00 01 00 00                        
0000008

$ hexdump counter15.bin 
0000000 0f 00 00 00 00 01 00 00                        
0000008
  \end{minted}
  \caption{Interpretatie van de gedownloade geheugenplaatsen.}
  \label{lst:node-capture-hexdump}
\end{listing}


%!TEX root=masterproef.tex
\section{Gerelateerd onderzoek}
\label{section:related}

Het detecteren van inbraken komt dikwijls neer op het detecteren van abnormaal
gedrag of anomalie\"en. Er zijn verschillende manieren hoe normaal gedrag kan
gedefinieerd worden elk met hun voor- en nadelen, beperkingen en successen.
(\ref{subsection:anomaly}).

Met de komst van draadloze sensornetwerken moesten veel klassieke
beveiligingstechnieken herbekeken worden. De schaal waarop deze netwerken
opereren, maar vooral de fysieke toegankelijkheid, waren parameters die niet
als problemen ervaren werden in meer klassieke computernetwerken. Hierdoor
moesten fundamentele eigenschappen die niet langer eenvoudig technologisch
afgedekt konden worden, opnieuw gedefinieerd worden. Zo leiden de hoge graad
van distributie en de beperkte mogelijkheden tot onderlinge communicatie al
snel tot de concepten reputatie en vertrouwen (\ref{subsection:reputation}).

Anderzijds zal blijken dat een knoop op zich niet eenvoudig kan beslissen of
hij al dan niet een andere knoop vertrouwt. Knopen zullen - en dit is eigenlijk
een fundamentele eigenschap van draadloze sensornetwerken - moeten samenwerken.
De nood voor co\"operatieve algoritmen (\ref{subsection:cooperation}) is een
belangrijke volgende bouwsteen.

Hoe voeden we dit co\"operatief opgebouwd vertrouwen in een omgeving waar een
aanvaller fysieke toegang heeft tot elke knoop en zowel de vluchtige als de
programmageheugens kan benaderen en wijzigen? Een logische piste is om op zoek
te gaan naar een manier om de programmacode die op een knoop ge\"installeerd is
te valideren voordat deze aangeroepen wordt. Is het eigenlijk wel mogelijk om
aan software-attestatie (\ref{subsection:attestation}) te doen? Eigenlijk is
software-attestatie slechts een specifieke vorm van het nagaan van de
integriteit van een bepaald aspect van een knoop. In dit geval gaat het om de
inhoud van het geheugen. In het algemeen kan dit beschouwd worden als het
opsporen van anomalie\"en.

Naast detectiealgoritmen, zijn ook pogingen gedaan om allesomvattende
raamwerken te cre\"eren. Deze zouden het detecteren en verijdelen van aanvallen
makkelijker moeten maken. (\ref{subsection:frameworks}).

Enkele werken die een goed overzicht bieden van de stand van zaken met
betrekking tot inbraakdetectie in WSN zijn o.a. \citep{mishra2004intrusion} en
\citep{alrajeh2013intrusion}. Naast grote overlappingen met de hierna
gepresenteerde topics, bevatten ze nog andere voorbeelden en/of indelingen
omtrent deze materie.

%!TEX root=masterproef.tex

\subsection{Detecteren van anomalie\"en}
\label{subsection:anomaly}

Een anomalie is een afwijking van een normaal verloop van gebeurtenissen of
iets of iemands gedrag. Een knoop uit een DSN heeft een redelijk eenvoudig en
constante levensloop. Typisch zal een knoop op regelmatige tijdstippen
\emph{wakker worden}, waarden opmeten aan de hand van zijn sensoren en deze
waarden doorsturen naar een centrale locatie. Verder zal een knoop, als
onderdeel van het netwerk, tevens zulke waarden van andere knopen doorsturen.
Dit patroonmatige gedrag kan ge\"identificeerd worden en in een model verwerkt
worden. Op basis van zulk een model kan vervolgens nagegaan worden of de acties
van een knoop op een gegeven moment in lijn zijn met het model of dat er sprake
is van een anomalie.

Zulk een afwijking van het verwachte gedrag kan wijzen op veranderingen van
buitenaf. Deze kunnen op hun beurt veroorzaakt zijn door een aanval op het
netwerk. Aangezien we niet alle communicatie van en naar knopen kunnen
onderscheppen en eventuele aanvallen kunnen detecteren, kunnen
anomaliegebaseerde dectectiemechanismen helpen om op basis van neveneffecten
toch aanvallen te detecteren, of althans toch de gevolgen ervan.

\subsubsection*{Anomali\"en, afwijkingen, aberraties}
\label{subsubsection:outlier}

\citep{zhang2010outlier} is een excellent overzicht van methoden om anomali\"en,
afwijkingen of aberraties (\emph{outliers}) te detecteren in een reeks
van metingen. Het belicht enerzijds de fundamentele technieken die ter
beschikking staan om deze aberraties op te merken maar tracht ook een
classificatie en taxonomie op te stellen hiervoor.

De auteurs stellen dat het detecteren van afwijkingen behoort tot het domein
van \emph{datamining} en dat het in die context reeds uitvoerig onderzocht is,
evenals binnen disciplines as statistiek, machinaal leren, informatie
theorie \dots. Aangezien het kunnen uitsluiten van afwijkingen de verwerking
van de overblijvende meetresultaten sterk positief be\"invloed is het in het
kader van DSN uitermate interessant.

Toch kan eerder onderzoek opnieuw niet eenvoudig toegepast worden in het kader
van DSN. De beperkte middelen van de sensorknoop schrappen al veel van de
klassieke oplossingen die bv. gecentraliseerd werken. De overdaad aan
communicatie die nodig is om voldoende gegevens te centraliseren voor
verwerking is niet realistisch in het kader van een DSN. Ook zijn veel van de
algoritmen typisch niet ontwikkeld met de beperkte rekenmogelijkheden van
sensorknopen. De conclusie is dat er een balans moet gevonden worden tussen de
mogelijkheden van datamining algoritmen voor het detecteren van afwijkingen en
de verhouding van hun noden ten opzichte van de middelen van de knopen.

\subsubsection*{Neurale netwerken}
\label{subsubsection:neuralnetworks}

Wanneer men denkt aan het vastleggen van een patroon en het controleren of een
bepaalde situatie voldoet aan dat patroon wordt in informatiecakringen ook snel
verwezen naar neurale netwerken. Neurale netwerken kunnen immers
\emph{getraind} worden door middel van een aantal goede (en slechte)
voorbeelden, waarna nieuwe voorbeelden kunnen gecatalogeerd worden als ook goed
of slecht. De complexiteit van het bepalen van deze beslissing is typisch
redelijk eenvoudig en lijkt zich daarom uitermate goed te lenen voor het
detecteren van anomalie\"en door sensorknopen.

In \citep{ramesh2012wireless} volgen de auteurs deze denkpiste, maar stellen
tevens dat er betere methoden bestaan. Ze trachten twee specifieke aanvallen
het hoofd te bieden: DoS en passieve informatie vergaring en vergelijken
hierbij een aanpak op basis van een neuraal netwerk en hun eigen aanpak op
basis van encryptie op basis van symmetrische sleutels.

Ofschoon dat sommige van hun veronderstelling na\"ief zijn (zo baseren ze zich
op een gedeelde geheime sleutel van 8 bits), toont hun werk wel aan dat een
aanpak met neurale netwerken eenvoudig te realiseren is en een valabele piste
kan zijn om anomaliedetectie te doen.

\subsubsection*{Voorspellingen}
\label{subsubsection:predictions}

Waar neurale netwerken in staat zijn om op basis van voorbeelden een nieuwe
situatie te catalogeren, kan men aan de hand van een Markov model
voorspellingen doen over de toekomst.

Het is deze piste dat onderzocht wordt in \citep{zhijie2012intrusion}. Opnieuw
betreft het een poging om DoS aanvallen te detecteren. De bedoeling is dat
sensorknopen individueel bepalen of er een DoS aanval bezig is. Volgens de
auteurs is dit mogelijk aan de hand van een Markov model dat het netwerkverkeer
voorspelt.

Het model wordt zo geconstrueerd dat er een verband ontstaat tussen de toestand
van een knoop in relatie tot het tijdstip en de verwachtte hoeveelheid gegevens
die verstuurd zouden kunnen worden.

Het idee achter het artikel lijkt een mogelijke piste, maar omtrent veel
belangrijke details blijven echter zeer vaag, waardoor de volledige toedracht
van het algoritme niet eenduidig ingeschat kan worden. Zo wordt bv. nauwelijks
ingegaan op wat de toestand van een knoop juist bepaalt of hoe de
hoeveelheid gegevens die verstuurd kunnen worden wanneer een knoop zich in een
bepaalde toestand bevindt, bepaald wordt.

%!TEX root=masterproef.tex

\subsection{Reputatie en vertrouwen}
\label{subsection:reputation}

De probleemstelling dat knopen in het netwerk elkaar niet langer kunnen
vertrouwen, zette verschillende onderzoekers aan tot het zoeken naar
oplossingen gebaseerd op reputatie en vertrouwen.

\citep{ganeriwal2008reputation} beschrijft een architectuur gebaseerd op
observaties door knopen van de acties van andere knopen in het kader van acties
van zichzelf of derde knopen. Figuur \ref{fig:reputation-cooperation} toont de
situaties die beschouwd worden: in \ref{fig:reputation-cooperative-node} zal
een co\"operatieve knoop (C) alle boodschappen die via hem verzonden worden
door een zendende knoop (Z) effectief doorsturen naar een verdergelegen
ontvangende knoop (O). De verzender van de boodschap, alsook andere naburige
knopen (B) kunnen deze actie vaststellen. In
\ref{fig:reputation-uncooperative-node} daarentegen zal een een
niet-co\"operatieve knoop (NC) deze boodschappen niet verder versturen of zelfs
aanpassen.

\begin{figure}
\centering
\begin{subfigure}{.49\textwidth}
\centering
  \includegraphics[width=.8\linewidth]{./resources/cooperative.pdf}
  \caption{Co\"operatieve knoop}
  \label{fig:reputation-cooperative-node}
\end{subfigure}
\begin{subfigure}{.49\textwidth}
\centering
  \includegraphics[width=.8\linewidth]{./resources/non-cooperative.pdf}
  \caption{Niet-co\"operatieve knoop}
  \label{fig:reputation-uncooperative-node}
\end{subfigure}
\caption{Beschouwde situaties bij al dan niet co\"operatieve knopen}
\label{fig:reputation-cooperation}
\end{figure}

Op basis van deze situatie stellen de auteurs dat de reputatie van een knoop
kan weergegeven worden aan de hand van een beta distributie. Bijlage
\ref{appendix:reputation} bespreekt de mathematische onderbouw hiervan.

De auteurs vermelden zelf een zeer belangrijk probleem: omdat knopen constant
moeten luisteren naar de acties van naburige knopen, moeten zij constant actief
zijn. Dit is een zeer nadelig uitgangspunt voor systemen die typisch trachten
zuinig om te springen met hun energie.

Maar de architectuur heeft ook inherente problemen en laat kwaadwillige
partijen toe om - mits kennis van de parameters - net onder de radar te
opereren. We illustreren dit met de simulatie zoals deze uitgevoerd werd door
de auteurs.

De evolutie van een volledige co\"operatieve of volledige niet-co\"operatieve
knoop wordt weergegeven in figuur \ref{fig:reputation-paper}. Een eigenschap
van het algoritme is dat pas na een tiental (louter positieve) observaties een
knoop de drempelwaarde van vertrouwen overschrijdt.

\begin{figure}[ht]
\centering
\begin{subfigure}{.49\textwidth}
  \centering
  \includegraphics[width=.9\linewidth]{./resources/reputation-paper.pdf}
  \caption{Co\"operatieve en niet-co\"operatieve knopen}
  \label{fig:reputation-paper}
\end{subfigure}
\begin{subfigure}{.49\textwidth}
  \centering
  \includegraphics[width=.9\linewidth]{./resources/reputation-with-failure.pdf}
  \caption{Falende knopen (100 simulaties)}
  \label{fig:reputation-with-failure}
\end{subfigure}
\caption{Impact van falende knopen op evolutie van vertrouwen}
\label{fig:reputation-paper-with-failure}
\end{figure}

Deze eigenschap kan echter misbruikt worden zoals aangetoond wordt in figuur
\ref{fig:reputation-with-failure}. Stel dat een knoop $j$ te kampen heeft met
falende hardware, waardoor 5\% van zijn transmissies verloren gaan en daarom
ook niet opgemerkt kunnen worden door andere knopen.

We merken op dat deze knoop, zelfs met 5\% niet-co\"operatieve observaties, na
een twintigtal observaties toch boven de drempelwaarde uitkomt en door de
beschouwende knoop aanvaard wordt als betrouwbaar.

Vanuit een operationeel standpunt gezien is dit in eerste instantie een
positief effect. Indien een knoop \emph{slechts} 5\% faalt zal deze toch als
co\"operatief beschouwd worden en de goede werking van het netwerk niet
fundamenteel in het gedrang brengen - vanuit een inbraakdetectie-oogpunt gezien.

Maar stel dat deze 5\% niet-co\"operatieve acties geen falen zijn en dat de
doorgestuurde boodschappen niet verloren gaan, maar met opzet lichtjes
gewijzigd worden. 5\% kan een significante vertekening van metingen van een
netwerk betekenen en zo de werking van het hele netwerk ondermijnen.

Figuur \ref{fig:reputation-malicious} gaat slechts een kleine stap verder en
toont het effect van falende (of malafide) knopen die pas falingen vertonen
nadat ze het vertrouwen hebben gekregen van een knoop. We merken op dat nu zelfs
10\% falingen zeer lang het vertrouwen kunnen behouden.

\begin{figure}[ht]
 \centering
 \includegraphics[width=.5\linewidth]{./resources/reputation-malicious.pdf}
 \caption{Falende knopen met vertraging van 15 pakketten (100 simulaties)}
 \label{fig:reputation-malicious}
\end{figure}

Dit elementaire voorbeeld toont duidelijk aan dat het vaststellen van een
reputatie op basis van externe observaties een zeer delicaat onderwerp is dat
zeer gevoelig is voor manipulatie op basis van kennis van de interne
parameters. Dit laatste is dan weer net \'e\'en van d\'e problemen waar
draadloze sensornetwerken mee kampen omdat knopen vrij eenvoudig kunnen
weggenomen, ge\"inspecteerd, gewijzigd en teruggeplaatst worden.

%!TEX root=masterproef.tex
\subsection{Co\"operatieve algoritmen}
\label{subsection:cooperation}

Het detecteren van abnormaal gedrag, dat op zijn beurt een indicatie kan zijn
van een (poging tot) inbraak door \'e\'en knoop is \'e\'en ding, als netwerk
van knopen tot een consensus komen en met meer zekerheid een verdachte knoop
uitsluiten is een heel ander ding.

Een veel voorkomend onderwerp is dat van co\"operatie tussen knopen, waarbij in
overleg bepaald wordt of en welke andere knoop uitgesloten moet worden uit het
netwerk. In \citep{krontiris2009cooperative} wordt hiertoe eerst langs een
theoretische weg gezocht naar de nodige en voldoende voorwaarden voor
inbraakdetectie. Vervolgens wordt er een praktisch omkaderend algoritme
voorgesteld om op co\"operatieve manier aan inbraakdetectie te doen.

Zowel dit theoretische model als het praktische algoritme vormen een
interessante bron van informatie. Het theoretische model kan helpen bij het
analyseren van andere co\"operatieve oplossingen en het praktische algoritme
biedt een algemeen raamwerk voor het implementeren van co\"operatieve
strategie\"en.

Bijlage \ref{appendix:idp-cooperation} gaat in meer detail in op de
mathematische onderbouw van het zgn. \emph{Intrusion Detection Problem} (IDP).
Naast een theoretisch model wordt tevens een algoritme voorgesteld dat het
mogelijk maakt om op gedistribueerde manier samen te werken en tot een
beslissing te komen aangaande de aanwezigheid van een malafide knoop in het
netwerk.

Het algoritme is een betrekkelijk eenvoudig raamwerk voor een co\"operatieve
aanpak, waarbij knopen zelfstandig beslissen welke andere knopen ze verdenken
en vervolgens gezamenlijk, op een gedistribueerde manier, trachten tot een
consensus te komen welke van de verdachte knopen effectief de aanvaller is.

De kracht van dit raamwerk en het succes ervan hangt natuurlijk sterk af van de
lokale detectiemogelijkheden van de knopen en de accuraatheid hiervan.

\subsubsection*{Risico's}

Het voorbeeld in figuur \ref{fig:idp-examples-2} beslaat een zeer beperkte
scope en de voorwaarden van het IDP kunnen in praktijk niet geverifieerd
worden. We moeten voorzichtig zijn niet te snel conclusies te trekken die in
een ruimere situatie misschien een verkeerd beeld zouden kunnen opleveren.
Figuur \ref{fig:sinkhole-ripple} toont essentieel hetzelfde voorbeeld als dat
van \ref{fig:idp-examples-2}, maar nu met meer knopen rondom het initi\"ele
voorbeeld.

Het routeringalgoritme is gebaseerd op de totale kost van het pad naar het
basisstation en komt daarmee overeen met het MultiHopLQI routering algoritme
beschreven in o.a. \citep{krontiris2008launching}. In dit werk wordt ook de
zgn. \emph{Sinkhole Attack} voorgesteld. We nemen deze aanval als voorbeeld.

\begin{figure}[ht]
\centering
\begin{subfigure}{.49\textwidth}
  \centering
  \includegraphics[width=.8\linewidth]{./resources/sinkhole-before.pdf}
  \caption{Initi\"ele topologie, routes en kosten}
  \label{fig:sinkhole-ripple-1}
\end{subfigure}
\begin{subfigure}{.49\textwidth}
  \centering
  \includegraphics[width=.8\linewidth]{./resources/sinkhole-after.pdf}
  \caption{Knoop $d$ kondigt ``betere'' route aan}
  \label{fig:sinkhole-ripple-2}
\end{subfigure}
\caption{Voorbeeld van het theoretische risico dat kan leiden tot een verkeerde
identificatie van de echte aanvaller}
\label{fig:sinkhole-ripple}
\end{figure}

Stel dat knoop $d$ een \emph{Sinkhole Attack} uitvoert door een zeer lage kost
te adverteren. Hierdoor zal knoop $a$ geneigd zijn om zijn route aan te passen.
Hierdoor zal deze op zijn beurt een veel voordeligere route adverteren en
zullen ook knopen $b$ en $c$ hun route wijzigen en hun gegevens via knoop $a$
versturen.

Ten gevolge van deze route-updates is het mogelijk dat een lokale detector voor
de \emph{Sinkhole Attack} op knopen $a$ en $b$ in werking zal treden. Hierbij
kunnen de knopen alleen hun volledige buurt beschuldigen, omdat het niet
mogelijk is om te detecteren wie de valse boodschappen effectief verstuurd
heeft. Indien de aanvallende knoop $d$ nu ook selectief zijn naburige knoop $a$
beschuldigt, komen we tot dezelfde situatie als in figuur
\ref{fig:idp-examples-2}, echter nu met mogelijk een verkeerd
ge\"identificeerde aanvaller, omdat in deze situatie niet voldaan is aan de
voorwaarden van het IDP.

Dit voorbeeld is, net zoals de vele andere beschreven voorbeelden, uitermate
specifiek en dient louter ter illustratie van het fragiele karakter van een
co\"operatief algoritme. Desalniettemin bieden de concepten en het omkaderende
algoritme ge\"introduceerd in \citep{krontiris2009cooperative} een goed
uitgangspunt voor het beschrijven en implementeren van
inbraakdetectiemechanismen.

\subsubsection*{Groeperen}
\label{subsubsection:grouping}

Een andere aanpak van co\"operatieve algoritmen vertrekt van het groeperen van
knopen. Deze aanpak wordt toegepast door \citep{li2008group}. Groepering
gebeurt op basis van nabijheid en tracht sensoren te groeperen die door hun
locatie gelijkaardige meetwaarden zouden moeten opmeten. De auteurs stellen een
algoritme voor op basis van verschillen, een zgn. delta-algoritme.

Metingen van knopen kunnen nu binnen de groep met elkaar vergeleken worden, en
afwijkende resultaten (zie ook sectie \ref{subsection:anomaly}) kunnen op
statistische wijze beschouwd worden als abnormaal gedrag en op die manier
gerapporteerd worden.

%!TEX root=masterproef.tex
\subsection{Attesteren van software}
\label{subsection:attestation}

Het voorbeeld uit sectie \ref{section:node-capture} toonde al aan dat zelfs het
vluchtige geheugen van een knoop niet veilig is. Als een aanvaller in staat is
om ongemerkt de programma-code van een knoop te bekomen, alsook alle gegevens
die alleen tijdens uitvoering in het geheugen, dan kan deze aanvaller deze code
aanpassen zodat de werking ogenschijnlijk ongewijzigd is, maar dat hij toch
controle heeft over de werking en zo het hele netwerk kan be\"invloeden.

Een zeer logische onderzoeksvraag dient zich al snel aan: ``\emph{Is het
mogelijk om wijzigingen aan het programma van een knoop in het netwerk vast te
stellen?}''. Deze vraag wordt onderzocht binnen het domein van software
attestatie.

\subsubsection*{Werking}

Alle bestaande vormen van software attestatie maken gebruik van een protocol
gebaseerd op het challenge response principe. Als men de integriteit van een
knoop wil vast stellen, zal men aan deze knoop een verzoek sturen om een unieke
samenvatting te maken van zijn inhoud door middel van een cryptografische
hashfunctie, een \emph{checksum}.

De vaststeller beschikt zelf over een versie van de inhoud van de knoop en kan
dezelfde unieke samenvatting berekenen. Door in het initi\"ele verzoek een
\'e\'enmalig te gebruiken code mee te geven, een zgn. \emph{nonce}, en deze
deel te laten uitmaken van de inhoud, kunnen verschillende verzoeken telkens
met een ander, unieke samenvatting beantwoord worden en wordt kan deze
samenvatting niet op voorhand gekend en berekend worden. Figuur
\ref{fig:attestation-process} geeft een overzicht van de werking van software
attestatie en illustreert hoe een wijziging door een aanvaller zich propageert.

\begin{figure}
  \centering
  \includegraphics[width=0.9\linewidth]{resources/attestation-process.pdf}
  \caption{De werking van software attestatie: een aanvaller heeft een
  wijziging kunnen aanbrengen in de programma code op een knoop. Deze wijziging
  propageert zich in de \emph{checksum} en wordt door de vaststeller opgemerkt.}
  \label{fig:attestation-process}
\end{figure}

De inhoud waarvan een samenvatting gemaakt wordt is typisch de programma code
die op de knoop ge\"installeerd werd. Indien een aanvaller deze code kon
wijzigen, zou de samenvatting niet langer overeenkomen met die opgesteld door
de vaststeller en kan deze laatste besluiten om deze gewijzigde code niet te
vertrouwen en de knoop uit te sluiten.

\subsubsection*{Implementaties}

SWATT werd voorgesteld in \cite{seshadri2004swatt}. Het is een attestatie
procedure die een \emph{checksum} berekent over nagenoeg alle geheugenlocaties,
echter wel in willekeurige volgorde. Anderzijds houdt SWATT ook rekening houdt
met de tijd die de attestatie routine op de knoop nodig heeft om de
\emph{checksum} te berekenen. Indien een aanvaller code zou toevoegen om de
werking van de attestatie routine te verstoren, zou op te merken zijn in een
vertraging.

Met SCUBA in \cite{seshadri2006scuba} en SAKE in \cite{seshadri2008sake} werd
verder gebouwd op de SWATT techniek met het oog op een beveiligde distributie
van programma code en het veilig uitwisselen van sleutels. Samen met SCUBA en
SAKE werd ook \emph{Indisputable Code Execution} of ICE ge\"introduceerd. Daar
waar SWATT gericht is op inhoudelijke integriteit, voegt ICE hieraan ook de
garantie van een niet aangetaste uitvoering van programma's aan toe en laat het
toe om beperkte regio's van het geheugen te benaderen.

Op deze manier kan nu bovenop de attestatie van het geheugen van een knoop, nu
ook functionaliteit aangeroepen worden, waarvan de werking ook gegarandeerd
veilig is. Het installeren van nieuwe code en het uitwisselen van gedeelde
geheimen wordt op die manier mogelijk.

ICE realiseert dit door een \emph{checksum} te berekenen over de geheugenregio
waar de attestatie routine zich bevindt, alsook over de regio waar het uit te
voeren programma staat en van de staat van de processor. Hierdoor ontstaat er
een garantie dat de attestatie correct verloopt, dat het uit te voeren
programma geen onbekende code bevat en dat de omgeving waarin de attestatie en
het programma uitgevoerd worden gegarandeerd niet aangetast kan worden.

Een belangrijke eigenschap van de ICE techniek is dat de attestatie routine de
processor in een emph{veilige} staat brengt door geen interrupts toe te laten.
Hierdoor kan de werking van de attestatie routine niet onderbroken en gewijzigd
worden. Na correcte attestatie zal het geattesteerde programma in dezelfde
veilige omstandigheden als de ICE routine uitgevoerd worden.

\subsubsection*{Evaluatie}

De beweringen rond SWATT en ICE werden in \cite{castelluccia2009difficulty}
onder de loep genomen en verschillende manieren om deze vormen van
integriteits-controle te omzeilen werden voorgesteld. Ondanks het feit dat
verschillende interessante aspecten van de attestatie technieken werden
belicht, werden te snel veronderstellingen rond beide implementaties gemaakt en
werd in \cite{perrig2010refutation} een weerwoord gegeven.

Desalniettemin zijn de ontwijkingstechnieken die voorgesteld werden zeer
interessante voorbeelden van de mogelijkheden die een aanvaller heeft tegen
software attestatie. Het feit dat de op het eerste zicht inderdaad valabele
aanvallen toch nog fouten bevatten, biedt dan weer een ander interessant beeld
op de kwaliteiten van de voorgestelde technieken. We belichten ze in de
volgende paragrafen als inspiratiebron.

De fundamentele manier om de attestatie code te omzeilen bestaat er in om de
opgevraagde geheugenadressen te controleren en indien ze verwijzen naar
plaatsen waar zich niet-originele code bevindt, deze te herschrijven naar
adressen waar de originele code zich bevindt.

Aangezien het merendeel van het programma geheugen op een knoop typisch leeg
is, kan de aanvaller zijn benodigde code verbergen in zo'n stuk leeg geheugen.
Mits zorgvuldige keuze van deze locatie, kan het controleren van en verwijzen
naar een andere locatie zich beperken tot de manipulatie van \'e\'en enkele bit
in het adres. Deze techniek wordt ook wel een geheugen schaduwende aanval
genoemd.

Om het probleem van leeg programma geheugen en de bijhorende uitnodiging aan
het adres van de aanvaller om zich eenvoudig te kunnen verschuilen, aan te
pakken, stelelen o.a. \cite{yang2007distributed,seshadri2008sake} voor om dit
geheugen voor dat het op een knoop wordt geplaatst, op te vullen met
willekeurige waarden. Op deze manier heeft de aanvaller geen vrije ruimte om
zijn code in te plaatsen.

Ofschoon deze willekeurige data inderdaad zo kan opgesteld worden dat ze niet
kan verkleind worden, kan dit niet gegarandeerd worden van de eigenlijke
programma code. Deze kan typisch wel nog verkleind worden en in die vorm
opgeslagen worden, waardoor er mogelijk voldoende ruimte vrijkomt voor de code
van de aanvaller. Op het ogenblik van attestatie kan deze oorspronkelijke code
dan, indien nodig terug hersteld worden.

Indien deze eenvoudige technieken toch niet voldoende ruimte zouden bieden, kan
er nog altijd gekeken worden naar het data geheugen. We merken immers op dat
nagenoeg alle vormen van attestatie alleen toegepast worden op het programma
geheugen. Het data geheugen is immers te veranderlijk en kan niet volledig
gekend zijn door de vaststeller.

Hierdoor wordt dit data geheugen natuurlijk het volgende mogelijke
aandachtspunt voor de aanvaller. Ondanks het feit dat ook op een \mcu het data
geheugen veelal niet kan uitgevoerd worden, blijft het mogelijk om programma
code in het data geheugen op te slagen en te kopi\"eren naar het programma
geheugen.

De \emph{rootkit} voorgesteld in \cite{castelluccia2009difficulty} hanteert dit
principe. Door middel van de \emph{Return Operation Programming} (ROP)
techniek, o.a. beschreven in \cite{prandini2012return}, kan een aanvaller met
eerste haak bij aanvang van de attestatie code in het programma geheugen, zijn
eigen rootkit uit het programma geheugen laten verwijderen. Na deze operatie is
het programma geheugen opnieuw intact en zal de originele attestatie routine
een positief resultaat opleveren. Maar de eerste haak heeft er ook voor gezorgd
dat de bij terugkeer uit de attestatie routine een tweede haak geplaatst is die
op zijn beurt de rootkit en de initi\"ele haak opnieuw door middel van ROP
instructies installeert. Figuur \ref{fig:attestation-rootkit} toont deze
werking.

\begin{figure}
  \centering
  \includegraphics[width=0.9\linewidth]{resources/attestation-rootkit.pdf}
  \caption{De werking van een attestatie ontwijkende rootkit.}
  \label{fig:attestation-rootkit}
\end{figure}

Het verbergen van de rootkit en het herstellen van het programma geheugen in
zijn oorspronkelijke staat blijkt slechts een overhead van ongeveer 0.3\% op te
leveren t.o.v. bv. de SWATT attestatie techniek voorgesteld in
\cite{seshadri2004swatt}. Deze techniek controleert tevens de tijd dat de
attestatie routine nodig had om de checksum te berekenen. Indien deze te lang
duurt dan schrijft SWATT dit toe aan de overhead ge\"introduceerd door
mogelijke kwaadaardige code. Een verhoging met 0.3\% is mogelijk te weinig om
tot deze conclusie te komen.

Deze aanval richt zich nu louter op de attestatie routine, maar zoals in
\cite{perrig2010refutation} aangegeven wordt, it SWATT slechts een deel van een
volledige software attestatie en focust zich op het effectief attesteren van
code in het geheugen, niet op de omringende context. In een volledige
opstelling zou een attestatie procedure de het terugkeer adres op de stack mee
kunnen nemen in de attestatie.

\cite{castelluccia2009difficulty} beschrijven zelf enkele mogelijke pistes
waarmee een attestatie routine zichzelf zou kunnen beschermen tegen een
dergelijke rootkit. Een eerste zou op het einde van zijn implementatie het data
geheugen volledig leeg kunnen maken en vervolgens, zonder een terugkeer
operatie uit te voeren, waardoor de tweede haak vermeden wordt, de knoop te
herstarten. Het verwijderen van alle gegevens en het herstarten van een knoop
bij elk attestatie verzoek, kan afhankelijk van de functionaliteit die de knoop
aanbiedt, in de meeste gevallen niet wenselijk zijn. Dit euvel kan eventueel
wel ondervangen worden door het wegschrijven van deze gegevens naar een EEPROM.

Een andere oplossingsstrategie zou kunnen liggen in het attesteren van het data
geheugen. Dit moet dan wel op op het zelfde ogenblik gebeuren als de attestatie
van het programma geheugen, zodat de rootkit niet in staat is om zichzelf heen
en weer te kopi\"eren tussen de twee afzonderlijke attestaties. Zelfs al wordt
er willekeurig telkens uit het ene of het andere geheugen gelezen, dan nog kan
de rootkit zich nog steeds verplaatsen tussen de twee geheugens en moet deze
dat dan zelfs maar gemiddeld om de twee lees operaties doen. Hierdoor zal de
overhead zelfs gedeeld door twee worden, waardoor de impact ervan nog
moeilijker te detecteren wordt.

Tot slot is, zoals eerder reeds vermeld, het attesteren van het werkgeheugen op
zich reeds een zeer moeilijk gegeven, door de onvoorspelbare inhoud ervan.
Idealiter zou de vaststeller ook de inhoud van het volledige data geheugen
moeten kennen. Enerzijds bevat dit geheugen registers enz., welke volledig
onvoorspelbaar zijn en dus uitgesloten zouden moeten worden. Anderzijds bevat
het geheugen gegevens van de actieve processen. Deze zijn typisch afhankelijk
van opgemeten waarden en/of communicatie met andere knopen en dus per definitie
onvoorspelbaar voor de vaststeller.

Opnieuw zou het leegmaken van het data geheugen een voorspelbaar resultaat
opleveren voor de vaststeller, maar dit is dan evengoed een voorspelbaar
resultaat voor de aanvaller en kunnen opnieuw geheugen schaduwende technieken
gebruikt worden.

Naast SWATT werd in \cite{castelluccia2009difficulty} ook een
aanvalsmogelijkheid tegen ICE voorgesteld. Het \emph{checksum} algoritme van
ICE is zo geconstrueerd dat het niet mogelijk is om bij elke geheugentoegang na
te gaan of er een doorverwijzing moet gebeuren of niet. Maar het is wel
mogelijk om een consequentie bit-wijziging te doen, zonder voorafgaande test.

Algoritme \ref{alg:attestation-ice} toont het \emph{checksum} algoritme. De
eigenlijke berekening vanaf regel 4 bestaat uit een strikte afwisseling van
16 bit optellingen zonder overdracht en XOR operaties ($\oplus$).

\begin{algorithm}
\begin{algorithmic}[1]
  \Require{y, het aantal iteraties dat de verificatie routine uitvoert}
  \For{$l = y \: to \: 0$}
    \Let{$x$}{$x + (x^2 \vee 5) mod 2^{16}$} \Comment{T functie voor $0 < x < 2^{16}$}
    \Let{$daddr$}{$(daddr \oplus) \wedge MASK) + code\_start$} \Comment{adres gebaseerd op $x$.}
    \Let{$C_j$}{$C_j + PC$}   \Comment{Program Counter}
    \Let{$C_j$}{$C_j \oplus mem[daddr]$}  \Comment{het willekeurige geheugenadres}
    \Let{$C_j$}{$C_j + l$}
    \Let{$C_j$}{$C_j \oplus C_{j-1}$}
    \Let{$C_j$}{$C_j + x$}
    \Let{$C_j$}{$C_j \oplus daddr$}
    \Let{$C_j$}{$C_j + C_{j-2}$}
    \Let{$C_j$}{$C_j \oplus SR$}    \Comment{Status register}
    \Let{$C_j$}{\Call{rotate\_left}{$C_j$}}
    \Let{j}{$(j+1)\: mod \: 10$}
  \EndFor
\end{algorithmic}
\caption{ICE pseudo-code\label{alg:attestation-ice}}
\end{algorithm}

Een mogelijke aanval op ICE bestaat er in om twee wijzigingen aan te brengen
die elkaar opheffen, waardoor een zelfde checksum berekent wordt, maar er toch
iets anders berekend wordt. Praktisch is het mogelijk om de meest
betekenisvolle bit van de Program Counter en van de waarde van de opgehaalde
geheugenlocatie te wisselen. Door de opeenvolging van de optelling en de XOR
operatie zullen deze elkaar opheffen, zoals getoond wordt in de vergelijkingen
\ref{eq:attestation-ice} en \ref{eq:attestation-ice-bitflip}, waarin een
voorbeeld wordt gegeven met 8 bits argumenten.

\begin{equation} \label{eq:attestation-ice}
\begin{array}{cccccccccc}
       & c_{j-1}    & 1 &	0 &	1 &	0 &	1 &	1 &	1 &	0 \\
+	     & PC	        & 0	& 1	& 1	& 0	& 1	& 0	& 1	& 1 \\
\cline{1-10}
       &            &	0	& 0	& 0	& 1	& 1	& 0	& 0	& 1 \\
\oplus &	mem[addr]	& 0	& 1	& 1 &	0	& 1	& 1	& 0	& 1 \\
\cline{1-10}
       &            &	0	& 1	& 1	& 1	& 0	& 1	& 0	& 0 \\
\end{array}
\end{equation}

\begin{equation} \label{eq:attestation-ice-bitflip}
\begin{array}{cccccccccc}
       & c_{j-1}    & 1 &	0 &	1 &	0 &	1 &	1 &	1 &	0 \\
+	     & PC	        & \bm{1}	& 1	& 1	& 0	& 1	& 0	& 1	& 1 \\
\cline{1-10}
       &            &	\bm{1}	& 0	& 0	& 1	& 1	& 0	& 0	& 1 \\
\oplus &	mem[addr]	& \bm{1}	& 1	& 1 &	0	& 1	& 1	& 0	& 1 \\
\cline{1-10}
       &            &	0	& 1	& 1	& 1	& 0	& 1	& 0	& 0 \\
\end{array}
\end{equation}

Het resultaat van deze minimale aanpassingen is dat er zich een situatie
voordoet zoals weergegeven in figuur \ref{fig:attestation-ice-copy}, waarbij de
aanvaller zijn eigen aangepaste kopie van de ICE routine kan laten uitvoeren,
terwijl de geattesteerde regio zich elders bevindt, waardoor de routine niet
meer zelf-attesterend is en zijn beginsel verliest.

\begin{figure}
  \centering
  \includegraphics[width=0.9\linewidth]{resources/attestation-ice-copy.pdf}
  \caption{De legitieme ICE routine is opgeslagen op adres 0x9100 en een
  aangepaste kopie is opgeslagen op adres 0x1100. Deze twee adressen
  verschillen slechts in hun meest betekenisvolle bit. De aanvaller kan zijn
  aangepaste code gebruiken, maar nog steeds slagen voor de attestatie.}
  \label{fig:attestation-ice-copy}
\end{figure}

In \cite{perrig2010refutation} bevestigen de auteurs van ICE dat er fouten zijn
geslopen in de finale definitie van ICE en dat ze opportuniteiten hebben laten
liggen om deze aanval tegen te gaan.

\subsubsection*{Conclusies en gevolgen}

Uit voorgaande paragrafen kunnen we vaststellen dat het op een eenvoudige \mcu
zeer moeilijk maar mogelijk moet zijn om een sluitende oplossing voor software
attestatie uit te realiseren. Er zijn echter veel mogelijkheden waarbij de code
van een aanvaller zich steeds tussen de verschillende stappen in de attestatie
procedure kan wringen. Zelfs indien de vaststeller rekening houdt met de tijd
die de knoop nodig heeft om de attestatie te voltooien, zijn er technieken die
snel genoeg zijn om ook hier binnen de aannemelijke grenzen te vallen. Maar al
deze ingrepen zijn zeer complex en vragen een perfecte voorbereiding.

Aan de andere kant kan voor bijna elk van deze aanvallen wel een aanpassing
toegevoegd worden aan een bestaande attestatie, zodat de aanval verijdeld kan
worden. Zoals met veel jonge beveiligings-gerelateerde wetenschappen is ook
hier het kat en muis nog lang niet gedaan en is er nog ruimte voor verder
onderzoek.

We kunnen concluderen dat software attestatie op een eenvoudige \mcu mogelijk
is, maar met de nodige omzichtigheid moet ge\"implementeerd worden.

\subsubsection*{Potentieel}

-> AMI situatie \cite{lemay2012cumulative}

\TODO

%!TEX root=masterproef.tex

\subsection{Raamwerken voor detectie}
\label{subsection:frameworks}

\TODO

\TODO \cite{valero2012di} -> Di-Sec

\TODO \cite{zhang2000intrusion} -> IDS architecture

\TODO \cite{kachirski2003effective} -> agents, not homogenous deployed (-> customized deployment)

\TODO \cite{krontiris2008lidea} -> LIDeA framework as basis for other kontiris' work

\TODO \cite{huang1999large} -> hiierarchical framework local/global



