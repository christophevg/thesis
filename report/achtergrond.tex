%!TEX root=masterproef.tex
\chapter{Achtergrond}
\label{chapter:achtergrond}

In dit hoofdstuk wordt het kader geschetst waarbinnen deze thesis op zoek gaat
naar antwoorden. Enerzijds wordt in sectie \ref{section:landscape} het
landschap van draadloze sensornetwerken in kaart gebracht: wat typeert en
onderscheid hen van andere netwerken? Waarom is het vaststellen van inbreuken
een belangrijk onderzoeksdomein?

Anderzijds wordt in sectie \ref{section:related} ingegaan op een groot aanbod
aan gerelateerd onderzoek. Een belangrijke doelstelling van deze thesis is het
in kaart brengen van de mogelijkheden en beperkingen betreffende reeds
beschreven methodes om inbreuken vast te stellen.

De verschillende beschreven methodes worden gecatalogeerd en gegroepeerd op
basis van verschillende eigenschappen. Op basis van dit overzicht wordt
vervolgens een inschatting gemaakt van de mogelijke dekking die kan bereikt
worden met de bestaande oplossingen.

Bij de beschrijving van de verschillende oplossingen zal tevens kritisch
nagegaan worden in hoeverre de oplossingen in een realistische situatie
effectief bijdragen tot het detecteren van inbreuken in het netwerk.

\section{Draadloze sensorennetwerken}
\label{section:landscape}

\TODO

\section{Gerelateerd onderzoek}
\label{section:related}

\TODO

\subsection{Reputatie en vertrouwen}

De probleemstelling dat knopen in het netwerk elkaar niet langer kunnen
vertrouwen, zette verschillende onderzoekers aan tot het zoeken naar
oplossingen gebaseerd op reputatie en vertrouwen.

\cite{ganeriwal2008reputation} beschrijft een architectuur gebaseerd op
observaties door knopen van de acties van andere knopen in het kader van acties
van zichzelf of derde knopen. Figuur \ref{fig:reputation-cooperation} toont de
situaties die beschouwd worden: in \ref{fig:reputation-cooperative-node} zal
een co\"operatieve knoop (C) alle boodschappen die via hem verzonden worden
door een zendende knoop (Z) effectief doorsturen naar een verder gelegen
ontvangende knoop (O). De verzender van de boodschap, alsook andere naburige
knopen (B) kunnen deze actie vaststellen. In
\ref{fig:reputation-uncooperative-node} daarentegen zal een een
niet-co\"operatieve knoop (NC) deze boodschappen niet verder versturen of zelfs
aanpassen.

\begin{figure}
\centering
\begin{subfigure}{.49\textwidth}
\centering
\[ \entrymodifiers={-=+++[o][F-]}
 \xymatrix@!=0.75pc {
  Z \ar[dr] & *{}       & *{} & *{} \\
  *{}       & C \ar[rr] & *{} & O   \\
  B         & *{}       & *{} & *{} \\
 }
\]
\caption{Co\"operatieve knoop}
\label{fig:reputation-cooperative-node}
\end{subfigure}
\begin{subfigure}{.49\textwidth}
\centering
\[ \entrymodifiers={-=+++[o][F-]}
 \xymatrix@!=0.75pc {
  Z \ar[dr] & *{}       & *{} & *{} \\
  *{}       & NC        & *{} & O   \\
  B         & *{}       & *{} & *{} \\
 }
\]
\caption{Niet-co\"operatieve knoop}
\label{fig:reputation-uncooperative-node}
\end{subfigure}
\caption{Beschouwde situaties bij al dan niet co\"operatieve knopen.}
\label{fig:reputation-cooperation}
\end{figure}

Gegeven knopen $i$ en $j$, met $\alpha_j$, het aantal observaties van acties van
knoop $j$ dat als co\"operatief werd beschouwd, en $\beta_j$, het aantal niet
co\"operatieve acties, toont men aan dat de reputatie van knoop $j$ wordt
weergegeven door een beta distributie van $\alpha_j$ en $\beta_j$:

\begin{equation} \label{eq:reputation-beta}
R_{ij} \sim Beta(\alpha_j+1, \beta_j+1)
\end{equation}

Van deze reputatie kan vervolgens een vertrouwen bepaald worden van knoop $i$
ten opzichte van knoop $j$ als volgt: 

\begin{equation} \label{eq:reputation-trust}
\begin{array}{rcl}
T_{ij} & = & E(R_{ij}) \\
       & = & E(Beta(\alpha_j+1, \beta_j+1)) \\
       & = & \frac{\alpha_j+1}{\alpha_j+\beta_j+2} \\
\end{array}
\end{equation}

$\alpha_j$ en $\beta_j$ evolueren doorheen de tijd. Hierbij dienen enerzijds
nieuwe observaties binnen afzonderlijke tijdspannes beschouwd te worden, maar
moet ook een wegingsfactor toegepast worden op de oude waarden om er voor te
zorgen dat een historisch opgebouwd beeld niet dominant blijft en nieuwe
wijzigingen in het gedrag overstemt. Gegeven $r$ het aantal co\"operatieve
observaties in een bepaalde tijdspanne en $s$ het aantal niet-co\"operatieve
observaties in diezelfde tijdspanne worden de nieuwe waarden voor $\alpha_j$ en
$\beta_j$ gegeven door:

\begin{equation} \label{eq:reputation-update-direct}
\begin{array}{rcl}
\alpha^{new}_j & = & (w_{age} \times \alpha_j) + r \\
\beta^{new}_j  & = & (w_{age} \times \beta_j) + s \\
\end{array}
\end{equation}

Hierbij is $w_{age}$ een factor ($< 1$) die zorgt voor een afname van de
belangrijkheid van de oudere informatie.

Naast deze eigen directe observaties, kunnen ook indirecte observaties door
naburige knopen in beschouwing genomen worden. Voor zo'n naburige knoop, $k$,
zal een knoop $i$ eveneens een vertrouwen $T_{ik}$ kunnen bepalen op basis van
$\alpha_k$ en $\beta_k$. Knoop $k$ kan vervolgens zijn eigen informatie met
betrekking tot de reputatie van knoop $j$ kenbaar maken als $\alpha^k_j$ en
$\beta^k_j$. Knoop $i$ kan vervolgens zijn parameters bijwerken als volgt:

\begin{equation} \label{eq:reputation-update-indirect}
\begin{array}{rrcl}
& \alpha^{new}_j & = & \alpha_j + ( w^k_{rep} \times \alpha^k_j ) \\
& \beta^{new}_j  & = & \beta_j  + ( w^k_{rep} \times \beta^k_j )  \\
met \\
& w^k_{rep}      & = & \frac{2 \alpha_k}{(\beta_k+2) (\alpha^k_j+\beta^k_j+2)+2 \alpha_k} \\
\end{array}
\end{equation}

De factor $w^k_{rep}$ zorgt er voor dat de opname van indirecte informatie van
knoop $k$ in verhouding tot zijn reputatie zal gebeuren.

Enkele bijkomende regels beschermen tegen typische problemen gerelateerd aan
deze aanpak: een knoop accepteert slechts indirecte informatie van een andere,
indien deze knoop zelf als vertrouwd wordt beschouwd. Hierbij wordt een
drempelwaarde ($TH_{SHI}$) gehanteerd. Verder wordt enkel positieve informatie
uitgewisseld, om negatieve be\"invloeding te vermijden. Tot slot wordt tevens
alleen directe informatie uitgewisseld, om de onafhankelijkheid van de
informatie te garanderen.

De auteurs vermelden zelf een zeer belangrijk probleem: omdat knopen constant
moeten luisteren naar de acties van naburige knopen, moeten zij constant actief
zijn. Dit is een zeer nadelig uitgangspunt voor systemen die typisch trachten
zuinig om te springen met hun energie.

Maar de architectuur heeft ook inherente problemen en laat kwaadwillige
partijen toe om - mits kennis van de parameters - net onder de radar te
opereren. We illustreren dit met de simulatie zoals deze uitgevoerd werd door
de auteurs.

De evolutie van een volledige co\"operatieve of volledige niet-co\"operatieve
knoop wordt weergegeven in figuur \ref{fig:reputation-paper}. Een eigenschap
van het algoritme is dat pas na een tiental (louter positieve) observaties een
knoop de drempelwaarde van vertrouwen overschrijdt.

\begin{figure}[h]
\centering
\begin{subfigure}{.49\textwidth}
  \centering
  \includegraphics[width=.9\linewidth]{./resources/reputation-paper.eps}
  \caption{Co\"operatieve en niet-co\"operatieve knopen}
  \label{fig:reputation-paper}
\end{subfigure}
\begin{subfigure}{.49\textwidth}
  \centering
  \includegraphics[width=.9\linewidth]{./resources/reputation-with-failure.eps}
  \caption{Falende knopen (100 simulaties)}
  \label{fig:reputation-with-failure}
\end{subfigure}
\caption{Impact van falende knopen op evolutie van vertrouwen.}
\label{fig:reputation-paper-with-failure}
\end{figure}

Deze eigenschap kan echter misbruikt worden zoals aangetoond wordt in figuur
\ref{fig:reputation-with-failure}. Stel dat een knoop $j$ te kampen heeft met
falende hardware, waardoor 5\% van zijn transmissies verloren gaan en daarom
ook niet opgemerkt kunnen worden door andere knopen.

We merken op dat deze knoop, zelfs met 5\% niet-co\"operatieve observaties, na
een twintigtal observaties toch boven de drempelwaarde uitkomt en door de
beschouwende knoop aanvaard wordt als betrouwbaar.

Vanuit een operationeel standpunt gezien is dit in eerste instantie een
positief effect. Indien een node \emph{slechts} 5\% faalt zal deze toch als
co\"operatief beschouwd worden en de goede werking van het netwerk niet
fundamenteel in het gedrang brengen - vanuit een inbraakdetectie oogpunt gezien.

Maar stel dat deze 5\% niet-co\"operatieve acties geen falen zijn en dat de
doorgestuurde boodschappen niet verloren gaan, maar met opzet lichtjes
gewijzigd worden. 5\% kan een significante vertekening van metingen van een
netwerk betekenen en zo de werking van het hele netwerk ondermijnen.

\TODO example of well-crafted malicious behavior

\TODO

\cite{krontiris2009cooperative,castelluccia2009difficulty,krauss2007detecting,seshadri2008sake,maerien2012famos,aschenbruck2012security,afzal2012difisec,yue2012novel,kuang2010snds,blilat2012wireless,ramesh2012wireless,valero2012di,perrig2004security,zhang2000intrusion,djenouri2005survey,yu2008framework,rassam2011novel,da2005decentralized,kachirski2003effective,li2008group,mishra2004intrusion,krontiris2008lidea,ioannis2007towards,soliman2012comparative,wang2011integrated,zhijie2012intrusion}
