%!TEX root=masterproef.tex

\subsection{Raamwerken voor detectie}
\label{subsection:frameworks}

Naast het onderzoek naar de verschillende autonome detectiealgoritmen, zijn er
ook onderzoekers actief op zoek naar manieren om meer holistische oplossingen
te bouwen die eerder het probleem als een geheel beschouwen in plaats van de
som van kleine, losse delen. In deze sectie bekijken we een aantal van deze
voorgestelde raamwerken.

\subsubsection*{Di-Sec}
\label{subsubsection:di-sec}

In \citep{valero2012di} komen de auteurs tot dezelfde conclusie als wat wij
hier enkele malen hebben aangehaald: er wordt veel gekeken naar
detailoplossingen, maar zelden wordt een algemene oplossing voorgesteld voor de
typische situatie van WSN. De auteurs zijn het verder ook eens met de stelling
dat het belangrijk is om te leren van deze detailoplossingen, om een eventueel
raamwerk af te stemmen op deze oplossingen, en gebruik te maken van de goede
eigenschappen ervan, bij de implementatie van het eigenlijke raamwerk.

De auteurs gaan er in hun redenering van uit dat, gegeven de eigenheid van het
draadloze medium, veel aanvallen zich richten op deze communicatievorm. Ze
defini\"eren daarom een eerste component, \ttt{COMM}, die zich toespitst op
communicatie en waarlangs alle gegevens passeren die verstuurd en/of ontvangen
worden. Daarnaast zijn er nog drie bijkomende componenten: \ttt{M-Core}, de
centrale controlemodule, \ttt{Sense}, de sensormodule en de \ttt{DDMs}, de
detectie- en verdedigingsmodules. Deze laatste zijn aanvalspecifieke modules
die ingevoegd kunnen worden in het raamwerk. Figuur
\ref{fig:di-sec-architecture} geeft een overzicht van de architectuur voor het
raamwerk, genaamd Di-Sec.

\begin{figure}[ht]
  \centering
  \includegraphics[width=0.8\linewidth]{resources/di-sec-architecture.pdf}
  \caption[Architectuur van Di-Sec]{Architectuur van Di-Sec (Bron:\citep{valero2012di})}
  \label{fig:di-sec-architecture}
\end{figure}

De drie modules vormen als het ware een API voor de \ttt{DDMs}. De manier
waarop \ttt{COMM} en \ttt{Sense} de toegang tot alle in- en uitvoer, zowel de
netwerkcommunicatie als de toegang tot de sensoren, hermetisch afsluiten, toont
aan dat Di-Sec een raamwerk is voor het ontwikkelen van sensorknopen.

Naast dit raamwerk, biedt Di-Sec ook een eigen taal aan om Di-Sec te gebruiken:
de \ttt{M-Core} controle taal (MCL). Deze laat toe om met een beperkte taal de
nodige sjablonen te maken waarmee nieuwe modules ontwikkeld kunnen worden en
alsook om de nodige aanpassingen te doen aan de configuratiebestanden.

\subsubsection*{Architectuur voor een sensorknoop}
\label{subsubsection:node-architecture}

Daar waar Di-Sec zich vooral focust op het aanbieden van een API voor het
ontwikkelen van detectiemodules, gaat \citep{zhang2000intrusion} een stap
verder en verrijkt de architectuur van een sensorknoop met verschillende
modules, gericht op de onderliggende processen van een IDS. Zo worden
beveiligde communicatiemodules voorzien, is er een co\"operatieve
detectiemodule, en voorzien zij globaal aggregerende modules om tot een
consensus te komen op het niveau van het netwerk. Deze architectuur is
schematisch weergegeven in figuur \ref{fig:node-architecture}.

\begin{figure}[ht]
  \centering
  \includegraphics[width=0.9\linewidth]{resources/node-architecture.pdf}
  \caption[Model voor een IDS op een sensorknoop]{Model voor een IDS op een
  sensorknoop (Bron:\citep{zhang2000intrusion})}
  \label{fig:node-architecture}
\end{figure}

De co\"operatieve modules voorzien bv. mogelijkheden om zekerheden te koppelen
aan beweringen over bepaalde gebeurtenissen. Op deze manier biedt de
architectuur een zeer specifiek aanbod aan functionaliteit, specifiek voor het
bouwen van een IDS.

Een gelijkaardige architectuur wordt voorgesteld in \citep{krontiris2008lidea}.
LIDeA is een uitgewerkte architectuur voor de idee\"en die al aan bod kwamen in
sectie \ref{subsection:cooperation}. Ze vormen het platform waarop het
co\"operatieve algoritme van \citep{krontiris2009cooperative} ge\"ent is.
Figuur \ref{fig:lidea-architecture} toont de LIDeA architectuur.

\begin{figure}[ht]
  \centering
  \includegraphics[width=0.9\linewidth]{resources/lidea-architecture.pdf}
  \caption[Architectuur van LIDeA]{Architectuur van LIDeA (Bron:\citep{krontiris2008lidea})}
  \label{fig:lidea-architecture}
\end{figure}
