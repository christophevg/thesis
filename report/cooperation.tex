%!TEX root=masterproef.tex
\subsection{Co\"operatieve algoritmen}
\label{subsection:cooperation}

Het detecteren van abnormaal gedrag, dat op zijn beurt een indicatie kan zijn
van een (poging tot) inbraak, door \'e\'en knoop is \'e\'en ding. Als netwerk
van knopen tot een consensus komen en met meer zekerheid een verdachte knoop
uitsluiten is een heel ander ding.

Een veel voorkomend onderwerp daarom is dat van co\"operatie tussen knopen om
in overleg te bepalen of en welke andere knoop uitgesloten moet worden uit het
netwerk. In \cite{krontiris2009cooperative} wordt hiertoe eerst langs een
theoretische weg gezocht naar de nodige en voldoende voorwaarden voor
inbraakdetectie. Vervolgens wordt er een praktisch omkaderend algoritme
voorgesteld om op co\"operatieve manier aan inbraak detectie te doen.

Zowel het theoretische model als het praktische algoritme vormen een
interessante bron van informatie. Het theoretische model kan helpen bij het
analyseren van andere co\"operatieve oplossingen en het praktische algoritme
biedt een algemeen raamwerk voor het implementeren van co\"operatieve
strategie\"en.

\subsubsection*{IDP}

Het theoretische model beschrijft het ``Intrusion Detection Problem'', IDP, aan
de hand van $S = \{ s_1, s_2, \dots s_n \}$, de set van sensoren in het netwerk,
$N(s)$, de set van buren van $s$ en $D(s)$, de set van knopen die door $s$
verdacht worden. Indien $|D(s)| = 1$, is de aanvaller ge\"identificeerd.

Enkele predicaten worden gedefinieerd als volgt: $source(q)$ geldt indien $q$
de aanvaller is, $honest(s) \iff \neg source(s)$, $expose_s(q) \iff D(s) = \{ q
\}$ ofwel $s$ verklaart dat $q$ de aanvaller is en $A(s) \iff D(s) \not= \{\}$
wat zoveel betekent als dat $s$ gealarmeerd is. De set van gealarmeerde buren
van een knoop $s$ wordt gedefinieerd als $AN(s) = \{ t | A(t) \wedge t \in N(s)
\}$ en de set van gealarmeerde buren die nuttig is voor een knoop $s$ wordt dan
uitgedrukt als $\tilde{AN}(q,s) = AN(q) \backslash \{s\}$.

Het IDP wordt vervolgens gedefinieerd als het vinden van een algoritme dat
voldoet aan de eigenschappen van correctheid: $\forall s \in S : honest(s)
\wedge expose_s(s') \implies A(s) \wedge source(s')$ en eindigheid: bij een
aanval zullen na een tijd alle eerlijke knopen in de gealarmeerde set een knoop
verdenken.

Twee condities worden voorgesteld: de ``Intrusion Detection Condition'' of IDC
en de ``Neighbourhood Conditions'' of NC. Indien aan minstens \'e\'en van deze
condities voldaan is, is het IDP oplosbaar.

De IDC wordt beschreven als $\forall p,q \in S : source(q) \implies
\tilde{AN}(p,q) \not= \tilde{AN}(q,p)$. Dit drukt uit dat geen enkele andere
knoop een zelfde gealarmeerde buurt kan hebben als de aanvaller.

De NC worden beschreven door ``\emph{alle buren van een aanvaller zijn
gealarmeerd}'' ($NC_1$) en ``\emph{indien twee of meer knopen verdacht zijn door
een meerderheid van knopen, hebben de eerlijke knopen niet-gealarmeerde buren}''
($NC_2$).

Figuur \ref{fig:idp-examples} illustreert het IDP en toepassing van IDC en NC
aan de hand van enkele voorbeelden:

\begin{figure}[ht]
\centering
\begin{subfigure}{.49\textwidth}
  \centering
\[ \entrymodifiers={}
 \xymatrix@!=0.75pc {
 *{}                                      & *{} & *-=++[o][F]{b} \ar@<0.7ex>[dll]\\
 *-=++[F]{a} \ar@<0.7ex>[urr] \ar@<0.7ex>[drr]& *{} & *{}                        \\
 *{}                                      & *{} & *-=++[o][F]{c} \ar@<0.7ex>[ull]\\
 }
\]
  \caption{}
  \label{fig:idp-examples-1}
\end{subfigure}
\begin{subfigure}{.49\textwidth}
  \centering
\[ \entrymodifiers={}
 \xymatrix@!=0.75pc {
 *{} & *{} & *{}                                      & *{} & *-=++[o][F]{b} \ar@<0.7ex>[dll] \ar[dd]\\
 *-=++[o][F]{d} \ar@<0.7ex>[rr] & *{} & *-=++[F]{a} \ar@<0.7ex>[ll] \ar@<0.7ex>[urr] \ar@<0.7ex>[drr]& *{} & *{} \\
 *{} & *{} & *{}                                      & *{} & *-=++[o][F]{c} \\
 }
\]
  \caption{}
  \label{fig:idp-examples-2}
\end{subfigure}
\caption{Voorbeelden van de toepassing van IDC en NC. Vierkante knopen zijn
aanvallers, omcirkelde knopen zijn gealarmeerd. $x \rightarrow y$ betekent dat
knoop $x$ knoop y verdenkt.}
\label{fig:idp-examples}
\end{figure}

De situatie in figuur \ref{fig:idp-examples-1} voldoet niet aan de IDC omdat
$\tilde{AN}(a,b) = \{c\} = \tilde{AN}(b,a)$. Maar in dit geval is wel voldaan
aan beide NC. Het IDP kan in dit geval opgelost worden aan de hand van een
deterministisch algoritme. Omdat er slechts \'e\'en knoop het hoogste aantal
verdenkingen op zijn naam heeft staan, kunnen knopen $b$ en $c$ eenvoudig
beslissen dat knoop $a$ de aanvaller is.

Figuur \ref{fig:idp-examples-2} toont een situatie waar de IDC wel voldaan is
want $\tilde{AN}(q,r) = \{s\} \not= \tilde{AN}(r,q) = \{\}$. Er zijn echter nu
twee knopen met een meerderheid aan verdenkingen: $a$ en $c$. Vanuit het
standpunt van knoop $b$ moet dus knoop $a$ of knoop $d$ valse aantijgingen
verspreiden. Als \'e\'en van deze knopen de aanvaller is, dan moet
$\tilde{AN}(a,d) \not= \tilde{AN}(d,a)$, anders is niet voldaan aan de IDC. Dit
impliceert tevens dat $\exists x : A(x) \wedge ( x \not\in N(a) \vee x \not\in
N(d) )$, ofwel dat er een knoop bestaat die gealarmeerd is, maar geen buur is
van de andere eerlijke knoop van de twee verdachte knopen. In dit geval zijn
knopen $b$ en $d$ inderdaad geen buren, maar beide verdenken knoop $a$.

\subsubsection*{Algoritme}
\label{subsubsection:cooperation-algorithm}

Naast een theoretisch model, introduceren \cite{krontiris2009cooperative}
tevens een algoritme dat kan dienen als raamwerk voor co\"operatieve
inbraakdetectie. Het algoritme bestaat uit vier tot vijf fasen: initialisatie,
stemming, publicatie van gebruikte sleutels, ontmaskeren van de aanvaller en
optioneel het inroepen van informatie uit de externe kring. Het is in essentie
een implementatie van het Guy Fawkes protocol, beschreven in
\cite{anderson1998new}, dat toelaat om een reeks van berichten te authenticeren
op basis van \'e\'en initi\"eel gedeelde sleutel.

Tijdens de initialisatie fase krijgt elke knoop een unieke sleutel, $K_l$. Aan
de hand van deze sleutel wordt een \'e\'en-richtingsketting van langte $l$
gemaakt van sleutels door het toepassen van bv. SHA1 \cite{rfc:3174} hashing
toe te passen: $\{K_0, K_1, \dots K_{l-1}\}$ waarbij $\forall k \in [1..l] :
K_{k-1} = SHA1(K_k)$. Daarnaast worden ook naburige knopen gezocht tot twee
knopen ver en wordt sleutel $K_0$ gecommuniceerd aan al deze buren. De
volledige initialisatie fase wordt verondersteld te gebeuren zonder
mogelijkheid tot inbraken.

Tijdens de stemming brengen alle knopen een stem uit van de vorm $m_v(s),$ $
MAC_{K_j}(m_v(s))$. $m_v(s)$ bestaat uit een lijst van knopen die door knoop
$s$ beschouwd worden als mogelijke aanvallers, of uitgedrukt aan de hand van
het theoretische model: $m_v(s) = D(s)$. De $MAC_{K_j}()$ functie is een zgn.
\emph{Message Authentication Code} \cite{rfc:2104} en wordt meestal berekend
door het toepassen van een \'e\'en-richtings-hashfunctie toe te passen op de
boodschap. Typisch wordt er aan de boodschap een unieke, wederzijds gekende
identificatie toegevoegd, zodat de ontvanger van een boodschap deze
handtekening ook kan berekenen en vergelijken met het origineel. In dit geval
wordt $K_j$ gebruikt, waarbij $j$ de volgende indexwaarde krijg uit lijst van
beschikbare sleutels.

Een dergelijke boodschap kan op ogenblik van verzending door geen enkele andere
knoop gevalideerd worden. De enige sleutel die zij tot op dat ogenblik kennen
is de vorige en de vorige is net het resultaat van een SHA1 operatie op de
volgende. Dit betekent dat ook een mogelijke aanvaller niet in staat is om de
boodschap te wijzigen.

Pas wanneer de knopen elkaars stemmen hebben ontvangen, wordt de gebruikte
sleutel vrijgegeven in de publicatie fase. Op dit ogenblik kunnen de knopen de
eerder verstuurde stemmen valideren. Ze kunnen eerst controleren of de
gepubliceerde sleutel inderdaad de juiste is. Immers, door het toepassen van
SHA1 op deze sleutel, moeten zij de huidige gekende sleutel bekomen. Na
validatie van de nieuwe sleutel, kan ook de boodschap gevalideerd worden.

Nu alle knopen de stemmen van alle knopen ontvangen hebben en zeker zijn dat de
stemmen authentiek zijn, kan met \'e\'enzelfde algoritme door alle knopen de
aanvaller bepaald worden tijdens de ontmaskering fase.

Het is echter mogelijk dat er meerdere knopen zijn met eenzelfde aantal stemmen
zijn, iets dat typisch overeenkomt met een gelijke samenstelling van
gealarmeerde buren, wanneer de IDC niet kan gerealiseerd worden. Onder
voorbehoud dat aan de NC wel voldaan wordt, kan door het inroepen van de niet
gealarmeerde buren van de gealarmeerde knopen. Deze zullen op hun beurt nagaan
of de knopen die verdacht worden, buren zijn en dan aangeven dat zij hen
\emph{niet} verdenken. Aan de hand van deze informatie kunnen de eerder
gealarmeerde knopen hun beslissing verder staven.

Het algoritme is een betrekkelijk eenvoudig raamwerk voor een co\"operatieve
aanpak, waarbij knopen lokaal beslissen welke andere knopen ze verdenken en
vervolgens gedistribueerd, gezamenlijk trachten tot een consensus te komen
welke van de de verdachte knopen effectief de aanvaller is.

De kracht van dit raamwerk en het succes ervan hangt natuurlijk sterk af van de
lokale detectie mogelijkheden van de knopen en de accuraatheid hiervan.

\subsubsection*{Risico's}

Het voorbeeld in figuur \ref{fig:idp-examples-2} beslaat een zeer beperkte
scope. We moeten voorzichtig zijn om hier niet te snel tot conclusies te komen
die in een ruimere situatie misschien een totaal verkeerd beeld zouden kunnen
opleveren. Figuur \ref{fig:sinkhole-ripple} toont essentieel het zelfde
voorbeeld als dat van \ref{fig:idp-examples-2}, maar nu met meer knopen rondom
het initi\"ele voorbeeld.

Het routering algoritme is gebaseerd op de totale kost van het pad naar het
basisstation en komt daarmee overeen met het MultiHopLQI routering algoritme
beschreven in o.a. \cite{krontiris2008launching}. In dit werk wordt ook de zgn.
\emph{Sinkhole Attack} voorgesteld. We nemen deze aanval als voorbeeld.

\begin{figure}[ht]
\centering
\begin{subfigure}{.49\textwidth}
  \centering
\[ \entrymodifiers={-=++[o][F]}
 \xymatrix@!=0.75pc {
*{} & e \ar@{->}[l]_{100} & *{} & f \ar@{->}[ll]_{50}   & *{} \\
*{} & *{} & *{} &  *{} & b \ar@{->}[lu]_{35} \ar@{.}[ld]_{20} \ar@{.}[dd]^{30}  \\
*{} & d \ar@{->}[uu]^{75} \ar@{.}[dd]_{75}  & *{} &    a \ar@{->}[lluu]_{80} \ar@{.}[lldd]^{80} \ar@{.}[ll]_{50} & *{} \\
*{} & *{} & *{} &  *{} & c \ar@{.}[lu]^{20} \ar@{->}[ld]^{35} \\
*{} & h \ar@{->}[l]^{105}  & *{} &  g \ar@{->}[ll]^{50}  & *{} \\
  }
\]
  \caption{Initi\"ele topologie, routes en kosten}
  \label{fig:sinkhole-ripple-1}
\end{subfigure}
\begin{subfigure}{.49\textwidth}
  \centering
\[ \entrymodifiers={-=++[o][F]}
 \xymatrix@!=0.75pc {
*{} & e \ar@{->}[l]_{100} & *{} & f \ar@{->}[ll]_{50}   & *{} \\
*{} & *{} & *{} &  *{} & b \ar@{.}[lu]_{35} \ar@{~>}[ld]_{20} \ar@{.}[dd]^{30}  \\
*{} & *-=++[F]{d} \ar@{~>}[l]^{50} \ar@{.}[uu]^{75} \ar@{.}[dd]_{75}  & *{} &    a \ar@{.}[lluu]_{80} \ar@{.}[lldd]^{80} \ar@{~>}[ll]_{50} & *{} \\
*{} & *{} & *{} &  *{} & c \ar@{~>}[lu]^{20} \ar@{.}[ld]^{35} \\
*{} & h \ar@{->}[l]^{105}  & *{} &  g \ar@{->}[ll]^{50}  & *{} \\
  }
\]
  \caption{Knoop $d$ kondigt ``betere'' route aan.}
  \label{fig:sinkhole-ripple-2}
\end{subfigure}
\caption{Voorbeeld van het theoretische risico dat kan leiden tot een verkeerde
identificatie van de echte aanvaller.}
\label{fig:sinkhole-ripple}
\end{figure}

Stel dat knoop $d$ een \emph{Sinkhole Attack} uitvoert door een zeer lage kost
te adverteren. Hierdoor zal knoop $a$ geneigd zijn om zijn route aan te passen.
Hierdoor zal deze op zijn beurt een veel voordeligere route adverteren en
zullen ook knopen $b$ en $c$ hun route wijzigen en nu hun gegevens via knoop
$a$ versturen.

Ten gevolge van deze route updates is het mogelijk dat een lokale detector voor
de \emph{Sinkhole Attack} op knopen $a$ en $b$ zal in werking treden. Hierbij
kunnen de knopen slechts hun volledige buurt beschuldigen, omdat het niet
mogelijk is om te detecteren wie de valse boodschappen effectief verstuurd
heeft. Indien de aanvallende knoop $d$ nu ook selectief zijn naburige knoop $a$
beschuldigd, bekomen we dezelfde situatie als in figuur
\ref{fig:idp-examples-2}, echter nu met mogelijk een verkeerd
ge\"identificeerde aanvaller.

Dit voorbeeld is, net zoals de vele andere beschreven voorbeelden, uitermate
specifiek en dient louter ter illustratie van het fragiele karakter van een
co\"operatief algoritme. Desalniettemin bieden de concepten en het omkaderende
algoritme ge\"introduceerd in \cite{krontiris2009cooperative} een goed
uitgangspunt om te hanteren bij het beschrijven en implementeren van inbraak
detectie mechanismen.

\subsubsection*{Groeperen}
\label{subsubsection:grouping}

Een andere aanpak van co\"operatieve algoritmen vertrekt van het groeperen van
knopen. Deze aanpak wordt toegepast door \cite{li2008group}. Groepering gebeurt
op basis van nabijheid en tracht sensoren te groeperen die door hun locatie
gelijkaardige meetwaarden zouden moeten opmeten. De auteurs stellen hiervoor
een dynamisch $\delta$-algoritme voor.

Metingen van knopen kunnen nu binnen de groep met elkaar vergeleken worden, en
op statistische wijze kunnen afwijkende resultaten (zie ook sectie
\ref{subsection:anomaly}) nu eventueel beschouwd worden als abnormaal gedrag en zo
gerapporteerd worden.
