%!TEX root=masterproef.tex

\section{Draadloze sensornetwerken}
\label{section:landscape}

``\emph{Waarom is het belangrijk om beveiliging van draadloze sensornetwerken
te bestuderen? Dit is toch al uitvoerig gedaan voor andere vormen van
netwerken!?}'' Op het eerste zicht is dit een zeer valabele opmerking. Reeds
sinds de late jaren '80 kennen we concepten als \emph{firewalls}, virussen,
wormen\dots Meer dan 30 jaar reeds wordt er onderzoek gedaan naar
computerbeveiliging en solide oplossingen zijn bedacht, uitgewerkt en
ge\"implementeerd. Wat houdt ons tegen om deze toe te passen op draadloze
sensornetwerken?

\subsection{Eigenschappen}

Ondanks het feit dat knopen in essentie kleine computers zijn en de
verbindingen die tussen hen tot stand komen een netwerk worden genoemd, eindigt
daar de vergelijking volledig. Draadloze sensoren en hun netwerken zijn een
wereld op zich, met zeer typerende eigenschappen, regels, mogelijkheden en
beperkingen.

Een draadloze sensor is inderdaad een kleine computer, maar de nadruk ligt hier
op \emph{kleine}. De rekenkracht van een knoop is slechts een fractie van deze
van een hedendaagse computer en ligt in de tientallen megahertz (MHz) - waar we
bij hedendaagse systemen spreken in termen van gigahertz (GHz). De reden is
evident: draadloze sensornetwerken zijn draadloos en worden ingezet in de meest
uiteenlopende en afgelegen situaties. Ze zijn daarom gedurende lange tijd
afhankelijk van een zelfde batterijvoeding, die dan ook optimaal benut moet
worden.

Ook het netwerk dat ze vormen vertoont typische eigenschappen die sterk
verschillen van de meer klassieke computernetwerken. Zo identificeert
\citep{blilat2012wireless} zes unieke eigenschappen van een WSN:

\begin{description}

  \item[Boomstructuur] De routering van de meeste draadloze sensornetwerken is
  \emph{gestructureerd als een boom}. De concepten co\"ordinator (of
  basisstation), router en eindknoop werden eerder, samen met hun structuur,
  ge\"identificeerd in sectie \ref{subsection:topologie}.

  \item[Aggregatie en redundantie] De gegevens door een knoop uit het netwerk
  verzameld, worden typisch niet op zich beschouwd, maar \emph{geaggregeerd}
  met deze van andere knopen. Dit wordt enerzijds gedaan om te compenseren voor
  effectieve aberraties in de metingen zelf, maar ook om de onzekerheid van de
  beschikbaarheid van knopen te ondervangen.

  \item[Vervangbaar] Knopen zijn typisch de goedkopere onderdelen van het
  netwerk en ze staan slechts ten dienst van het eigenlijke doel van het
  netwerk, het verzamelen van gegevens. Om die reden zijn ze eenvoudig
  vervangbaar en dit wordt als een inherente eigenschap gezien. Het netwerk
  \emph{tolereert storingen} ten gevolge van het wegvallen van knopen en vangt
  ze op door redundantie en aggregatie.

  \item[Verwerking in het netwerk] Om er voor te zorgen dat het netwerk zo min
  mogelijk belast wordt met het verzenden van gegevens, worden meetgegevens zo
  dicht mogelijk bij de oorspronkelijke knoop \emph{gefilterd en verwerkt}.

  \item[Uniformiteit] Een draadloos sensornetwerk bestaat uit knopen en niets
  anders. Elke knoop kan tegelijkertijd \emph{sensor} of \emph{router} zijn,
  wat resulteert, in combinatie met de voorgaande eigenschappen, in een
  reductie van het netwerkverkeer.

  \item[Energiebesparing] De werking van een knoop kent een typisch
  \emph{gefaseerd zendpatroon}: het verzamelen van meetgegevens, het ontvangen
  van gegevens van andere knopen en het doorsturen van geaggregeerde gegevens
  naar hogerliggende knopen. Hierdoor kan elke knoop zijn radio gedurende
  bepaalde periodes volledig afzetten om energie te besparen.

\end{description}

Dit beeld vinden we ook terug bij \citep{aschenbruck2012security} die de
situatie van draadloze sensornetwerken samenvat in drie zeer typische en
problematische bronnen van beveiligingsproblemen: 

\begin{itemize}[noitemsep, topsep=0pt, partopsep=0pt]

  \item{Beperkingen qua middelen en dan vooral de energievoorziening.}
  
  \item{Fysieke toegankelijkheid die leidt tot de mogelijkheid om knopen te
  veroveren.}

  \item{Verwerking van gegevens die reeds binnen het netwerk gebeurt, waardoor
  het bv. onmogelijk wordt om encryptie toe te passen tussen zender en
  ontvanger, waardoor tussenliggende verwerking eigenlijk uitgesloten wordt.}

\end{itemize}

\subsection{Inbraakdetectie}

\citep{zhang2000intrusion} geeft een zeer algemene, maar zeer correcte
definitie van inbraakdetectie:

\begin{quote}
\emph{Intrusion detection ... involves capturing audit data and reasoning about
the evidence in the data to determine whether the system is under attack.}
\end{quote}

Inbraakbeveiliging omvat twee essenti\"ele activiteiten: het vastleggen van
auditgegevens en het redeneren over deze gegevens. In het geval van draadloze
sensornetwerken moeten we het aspect \emph{systeem} op twee manieren
interpreteren: enerzijds als een knoop en anderzijds als het hele netwerk op
zich. Dit onderscheid bestaat ook in meer klassieke computernetwerken in de
vorm van inbraakdetectie voor \emph{systemen} en \emph{netwerken}.

In de context van draadloze sensornetwerken zal dit onderscheid zich echter
veel meer profileren omdat het netwerk hier geen centraal medium is. Daar waar
het netwerk in een klassieke opstelling typisch vanuit \'e\'en systeem kan
gecontroleerd worden door het afleiden van alle netwerkverkeer naar \'e\'en
enkele detectiemodule, is dit in het geval van een WSN niet mogelijk. Het
draadloze aspect maakt dat er geen controleerbare toegangswegen zijn naar elke
knoop en dat dus elke knoop letterlijk op zichzelf aangewezen is.

Samen met het wegvallen van een centrale inbraakdetectie, valt ook de centrale
bescherming op netwerkniveau weg. Binnen een WSN is het ook niet mogelijk om te
genieten van de bescherming van een \emph{firewall}. Er is geen alles omarmende
bescherming, niet op logisch of netwerkvlak, maar ook niet op fysiek vlak.
Aangezien draadloze sensornetwerken typisch open en bloot in de buitenwereld
ge\"installeerd worden, en dit voor langere periodes zonder aanwezigheid van
een eigenaar, kunnen ze ook fysiek benaderd worden door ieder die dat wil.
Zelfs de elementaire zekerheid van een fysieke bescherming bv. in een
datacenter, komt bij draadloze sensornetwerken volledig te vervallen.

Terecht stelt \citep{perrig2004security} daarom ook dat de eerste zorg omtrent
de beveiliging van draadloze sensornetwerken een veilige manier om de groep van
sensoren te beheren moet zijn. De manier waarop nieuwe knopen in het netwerk
worden opgenomen en de beveiliging van de onderlinge communicatie, is van
primordiaal belang. Voorkomen is beter dan genezen.

Maar we moeten ook realistisch zijn. Geen door mensenhanden gemaakt systeem is
feilloos en nagenoeg elk ge\"informatiseerd systeem zal met de nodige inzet en
moeite veroverd kunnen worden. Op dat ogenblik is het belangrijk dat er een
tweede verdedigingslinie is: inbraakdetectie. In het geval van WSN is dit
misschien nog meer prangend, omdat er niet langer een fysieke bescherming, noch
een centrale netwerkbescherming in de vorm van bv. een \emph{firewall} is.

\subsection{De tegenstander en zijn aanvallen}

Tijd om kennis te maken met de tegenstander. Wat is zijn doel? Welke middelen
heeft hij ter beschikking? Hoe kunnen we zijn aanvallen identificeren?

Identificatie van aanvallen wordt door \citep{zhang2000intrusion} gecatalogeerd
als \emph{misbruik} of \emph{anomalie}. Misbruik kan typisch gedetecteerd
worden aan de hand van patronen, terwijl voor het detecteren van anomalie\"en
er een patroon moet opgesteld worden en aberraties van dat patroon vastgesteld.
Vooral dit laatste is een delicaat aspect. Zo is het niet eenvoudig om
onderscheid te maken tussen een knoop die verplaatst werd, en tijdelijk
verkeerde routing informatie verspreid, en een knoop die veroverd werd en
kwaadwillig foutieve informatie uitstuurt.

\citep{aschenbruck2012security} identificeert vier fundamentele doelen die
nagestreefd worden door een aanvaller: (1) het manipuleren van gegevens, (2)
het afluisteren van communicatie, (3) het tegenhouden van gegevens en (4)
toegang verkrijgen tot het netwerk. Figuur \ref{fig:wsn-threat-analysis} geeft
een schematisch overzicht van deze dreigingsanalyse.

\begin{figure}[ht]
  \centering
  \includegraphics[width=0.95\linewidth]{resources/wsn-threat-analysis.pdf}
  \caption[Dreigingsanalyse van een WSN]{Dreigingsanalyse van een WSN (Bron:\citep{aschenbruck2012security})}
  \label{fig:wsn-threat-analysis}
\end{figure}

\vspace{-5mm}

\subsubsection*{Applicatie laag}

Een niveau dat hier nogal nadrukkelijk ontbreekt is dat van de applicatie.
Bovenop de verschillende netwerklagen van een sensorknoop, zal de sensor nog
typisch een niveau kennen dat specifiek is voor het desbetreffende WSN. Het
bevat de functionaliteit die het netwerk zijn bestaansreden geeft.

Ondanks de overvloed aan mogelijke aanvallen op de standaard netwerklagen, mag
het bestaan van de applicatielaag niet uit het oog verloren worden. Het
merendeel van aanvallen spitst zich net toe op fouten in deze laag. Deze kunnen
tevens dikwijls uitgebuit worden met normale vormen van communicatie. Zo hoeft
een aanvaller geen aanvallen op te zetten zoals de hoger vermelde voorbeelden,
als hij eenvoudig een legitieme boodschap kan zenden naar een gewone knoop en
een antwoord kan ontvangen met de informatie die hij wenst.

Op deze manier is er ook geen sprake van ``te detecteren malafide gedrag'' en
zal de aanval dikwijls onopgemerkt gebeuren. We denken hier aan klassieke
aanvallen zoals bv. \emph{buffer overflows}, waarbij andere gegevens uit het
geheugen worden gelezen dan bedoeld.
