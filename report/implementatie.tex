%!TEX root=masterproef.tex

\chapter{Implementatie}
\label{chapter:implementatie}

Voor deze thesis werd een prototype ge\"implementeerd van de generator. Hierbij
werd FOO-lang gedefinieerd tot op het niveau dat het mogelijk was om twee
realistische voorbeelden te beschrijven. Ook de generator werd uitgewerkt tot
het niveau dat het mogelijk was om de twee voorbeelden te genereren. Zowel de
voorbeelden als de implementatie van de taal en generator zijn zo uitgewerkt
dat ze als realistische referentie kunnen dienen en dat de resultaten het
potentieel waarborgen.

Het hoofdstuk wordt ingeleid met een korte sectie, \ref{section:devel-python},
over Python, de programmeertaal die werd gekozen voor de implementatie van het
prototype.

Daarna wordt FOO-lang in meer detail bekeken in sectie
\ref{section:devel-foo-lang}. Aan de hand van voorbeelden en de grammatica
introduceren we de taal. Een elementair voorbeeld wordt vervolgens als rode
draad doorheen het hoofdstuk gebruikt om de volledige generatie van een
FOO-lang beschrijving tot C programmacode te illustreren.

Sectie \ref{section:devel-codegen} belicht de generator met in hoofdzaak de
tweeledige hi\"erarchi van het SM en het CM. Ook het principe van
transformaties wordt kort samengevat en de onderliggende implementatie van het
\emph{visitor} patroon \citep{gamma1994design} wordt toegelicht.

De generator wordt vergezeld van gemeenschappelijke basisfunctionaliteit in de
vorm een softwarebibliotheek, genaamd FOO-lib. Sectie
\ref{section:devel-foo-lib} kadert dit in de context van de generator en de
taal.

Tot slot bekijken we het resultaat van de samengang van al deze componenten. In
sectie \ref{section:generation} bekijken we de gegenereerde code en zien we hoe
de generator de code structureert en vormgeeft.

%!TEX root=masterproef.tex

\section{Python}
\label{section:devel-python}

Als programmeertaal voor het prototype werd geopteerd voor Python, een
ge\"interpreteerde taal met dynamische typering die tevens verschillende
programmeerparadigma ondersteunt: imperatief, object-geori\"enteerd en
functioneel. Dit maakt het een zeer veelzijdige taal met veel mogelijkheden.

Python is volledig open in zijn structuur. Alles is toegankelijk en niets wordt
verborgen. Dit laat toe om elk aspect van een gegevensstructuur te manipuleren,
wat heel handig is, maar ook kan leiden tot onverwachte neveneffecten.

Alle functionaliteit, klassen of gewone functies, worden verzameld in een
\emph{module} en andere modules kunnen vervolgens deze functionaliteit
importeren. Door de volledige transparantie en dankzij introspectie, kan de
implementatie van een module zelfs dynamisch aangepast worden. Dit werd o.a.
toegepast voor het implementeren van het \emph{visitor}-patroon, in meer detail
besproken in bijlage \ref{appendix:visitor}.

Verder beschikt Python over een zeer rijke verzameling van kant-en-klare
modules, die toelaten om enkele basistaken vlot te implementeren. De
flexibiliteit en de mix van zowel imperatief als object-geori\"enteerd als
functioneel programmeren liet meermaals toe om bepaalde zaken op creatieve
manier te implementeren.

%!TEX root=masterproef.tex

\section{FOO-lang}
\label{section:devel-foo-lang}

De echt belangrijke taal in dit geval is niet Python, maar FOO-lang.
Codevoorbeeld \ref{lst:hello.foo} toont de implementatie van een elementair
voorbeeld in FOO-lang. Aan de hand van dit voorbeeld introduceren we de
typische bouwstenen van FOO-lang en doorlopen we het generatieproces.

\begin{listing}[ht]
  \inputminted[linenos,frame=lines,framesep=2mm,fontsize=\footnotesize]{js}{../src/foo-lang/examples/hello.foo}
  \vspace{-3mm}
  \caption{Elementair voorbeeld in FOO-lang: \ttt{hello.foo}}
  \label{lst:hello.foo}
\end{listing}

De code start op regel 6 met de declaratie van een \emph{module}. Een module is
een op zich staand geheel en zou bv. een detectiealgoritme kunnen zijn. Alles
wat volgt op de declaratie van de module, maakt er deel van uit.

Op regel 8 introduceren we een \emph{const}ante, \ttt{interval} en stellen die
gelijk aan \ttt{1000}. Hier zien we een eerste voorbeeld van het ontbreken van
expliciete typering in FOO-lang. Dankzij deductie van types zal in dit geval
het type van \ttt{interval} overeenkomen met een \emph{IntegerType}, omdat de
waarde \ttt{1000} gevormd is als een integer getal.

Ofschoon FOO-lang ge\"introduceerd wordt als DSL voor inbraakdetectie in WSN,
specificeert het zijn domein als dat van \emph{sensorknopen} of \emph{nodes}.
algoritmen met betrekking tot inbraakdetectie in WSN, hebben \'e\'en belangrijk
gemeenschappelijke entiteit en dat zijn de sensorknopen. Deze communiceren met
elkaar en op basis van die communicatie zijn zowat alle algoritmen opgebouwd.

Het raamwerk van de generator beheert het concept van een knoop of \emph{node}.
De algoritmen krijgen toegang tot deze knopen via een aantal functionele
constructies en ze kunnen de basisdefinitie van een knoop in het domein
uitbreiden met eigen eigenschappen. Dit gebeurt bv. op regel 10, waar (de knoop
van) het domein uitgebreid wordt met een eigenschap \ttt{sequence}. Deze
eigenschap wordt expliciet getypeerd als een \emph{byte} en krijgt als
initi\"ele waarde \ttt{0}.

FOO-lang is een \emph{functie}-geori\"enteerde taal die tracht om de functies
in de verschillende modules zo te organiseren dat de uitvoering ervan de \mcu
of de draadloze radio zo min mogelijk belast. Op regel 14 wordt een functie
gedefinieerd, genaamd \ttt{step}. Ze accepteert \'e\'en parameter, genaamd
\ttt{node}. We merken opnieuw op dat deze parameter niet getypeerd is.

De inhoud van de functie bestaat uit vertrouwde codeconstructies die bijna
gewone C-code kunnen zijn. Een conditie, een eigenschap, een
waardeverhoging\dots We zien hier onze eerder toegevoegde eigenschap,
\ttt{sequence}, terug opduiken.

Regel 20 brengt alle voorgaande definities samen in een
\emph{uitvoeringsstrategie}. FOO-lang tracht door middel van zijn syntax
leesbaar te zijn als een natuurlijke taal. Indien we regel 20 luidop lezen, kan
dit resulteren in: ``\emph{At every (passing of) interval with nodes do (the
function named) step.}''. En dat is exact wat deze regel definieert.

In deze ene regel zien we de lus over alle gekende knopen te voorschijn komen.
In alle algoritmen komt deze wel in \'e\'en of andere vorm terug. De lus is nu
echter geabstraheerd tot zijn functionele betekenis en onder controle van de
generator.

\subsection{Syntax en grammatica}
\label{subsection:devel-foo-lang-grammar}

Het voorgaande voorbeeld gebruikt slechts een kleine subset van de volledige
mogelijkheden van FOO-lang. De volledige grammatica van FOO-lang, zoals
gedefinieerd in het kader van dit prototype, is opgenomen in bijlage
\ref{appendix:foo-lang-grammar}. Deze bevat de \emph{Extended Backus-Naur Form}
(EBNF) die de taal eenduidig bepaalt. We bespreken hier kort NOG enkele andere
constructies.

\vspace{-3mm}

\subsubsection{Importeren van functionaliteit}

Het is mogelijk om externe functies te importeren. Dit gebeurt aan de hand van
de constructie \ttt{from ... import ... }, die een functie importeert vanuit
een module. Zo'n module kan een andere FOO-lang module zijn, of een extern
gedefinieerde functie uit een softwarebibliotheek. In
\ref{section:devel-foo-lib} introduceren we de FOO-lib, de standaard
softwarebibliotheek die de codegeneratie vervolledigt. Deze bevat tal van
functionaliteit die voorkomt in beschrijvingen van detectiealgoritmen.

\vspace{-3mm}

\subsubsection{Reageren op gebeurtenissen}

Naast het herhaaldelijk uitvoeren van een functie is het ook mogelijk om een
functie te koppelen aan een gebeurtenis in de context van het domein en de
knopen. In plaats van de \ttt{with ... do} constructie is het mogelijk om
v\'o\'or of n\'a (\ttt{before} en \ttt{after}) de uitvoering van een functie
een andere functie uit te voeren. Codevoorbeeld \ref{lst:foo-event_handler}
geeft een eenvoudig voorbeeld waarbij we ontvangen berichten tellen als deze
aan ons geadresseerd waren.

\begin{listing}[ht]
  \begin{minted}[linenos,frame=lines,framesep=2mm,fontsize=\footnotesize]{javascript}
after nodes receive do function(me, sender, from, hop, to, payload) {
  if(to == me) {
    me.msg_count++
  }
}
  \end{minted}
  \vspace{-5mm}
  \caption{Voorbeeld van het reageren op een gebeurtenis}
  \label{lst:foo-event_handler}
\end{listing}

In dit voorbeeld merken we ook op dat functies anoniem kunnen gedefinieerd
worden en niet louter eerst met een naam.

\vspace{-3mm}

\subsubsection{Verschillende situaties afhandelen}

Het \ttt{case statement} laat toe om eenzelfde expressie te evalueren in
verschillende situaties. Typisch gebruik voor deze constructie is het
analyseren van ontvangen gegevens. Codevoorbeeld \ref{lst:foo-case} illustreert dit.

\begin{listing}[ht]
  \begin{minted}[linenos,frame=lines,framesep=2mm,fontsize=\footnotesize]{javascript}
after nodes receive do function(me, sender, from, hop, to, payload) {
  case payload {
    contains([#marker1, value]) {
      sender.msg_sent1++
    }
    contains([#marker2, value]) {
      sender.msg_sent2++
    }
    else {
      sender.msg_other++
    }
  }
}
  \end{minted}
  \vspace{-5mm}
  \caption{Voorbeeld van het afhandelen van verschillende situaties}
  \label{lst:foo-case}
\end{listing}

In dit voorbeeld worden de ontvangen berichten geanalyseerd en gecatalogeerd op
basis van de inhoud. Als in een bericht \ttt{\#marker1} wordt gevonden wordt
\ttt{msg\_sent1} verhoogd, in het geval van \ttt{\#marker2}, \ttt{msg\_sent2}
en anders \ttt{msg\_other}.

Dit voorbeeld introduceert tevens nog drie andere belangrijke constructies: het
\emph{atom}, lijsten en patroonkoppeling.

\vspace{-3mm}

\subsubsection{Atomen}

\emph{Atomen} werden ontleend aan Erlang \citep{armstrong1993concurrent}. Ze
stellen een uniek herkenbare entiteit voor. Het is aan de generator om met hulp
van het platform en/of domein en/of doeltaal hiervoor een geschikte
voorstelling te vinden.

In het geval van dit prototype werden twee bytes voorzien om een unieke
identificatie te maken van delen in berichten. De generator kan verschillende
strategie\"en volgen en beslissen op basis van de verschillende atomen die hij
tegenkomt.

\vspace{-3mm}

\subsubsection{Lijsten}

Lijsten worden voor verschillende doeleinden gebruikt. Ze worden syntactisch
gespecificeerd door middel van vierkante haken. Zo worden ze meestal als een
letterlijke voorstelling van de lijst opgenomen. In het voorbeeld in listing
\ref{lst:foo-case} is de \ttt{payload} parameter zo'n lijst. Ook het argument
van de \ttt{contains} methode die toegepast wordt op \ttt{payload} is een
lijst. De parameter verwacht echter niet echt een lijst, maar een patroon. In
dit geval is de lijst een deel van het patroon.

\vspace{-3mm}

\subsubsection{Patroon koppeling}

Door middel van patronen kan er een kopping gemaakt worden tussen gegevens. In
het voorbeeld in codevoorbeeld \ref{lst:foo-case} accepteert de
\ttt{contains}-methode op een \emph{lijst} een patroon. Dit patroon is een
lijst en bestaat uit variabele en niet-variabele elementen. De
\ttt{contains}-methode zal trachten de niet-variabele elementen te herkennen in
de lijst en vervolgens bij succesvolle herkenning een koppeling te maken tussen
de variabele elementen en de waarden in de oorspronkelijke lijst.

\vspace{-3mm}

\subsubsection{Complexe types}

In codevoorbeeld \ref{lst:hello.foo} zagen we reeds kort een voorbeeld van
typering. De \ttt{sequence} eigenschap werd getypeerd als een \ttt{byte}. Naast
\ttt{byte} bestaan ook \ttt{integer}, \ttt{float}, \ttt{boolean} en
\ttt{timestamp} als standaard eenvoudige types.

Vergelijkbaar met lijsten is er ook het \emph{tuple} type. Dit is een lijst van
types en defini\"eert de types van de elementen van een lijst van vaste lengte.

Van alle types kan ook een veelvoud gedefinieerd worden door het toevoegen van
een sterretje (\ttt{*}) achter het type. Zo kunnen lijsten van een bepaald type
gedefinieerd worden. Gecombineerd met het \emph{tuple} type, ontstaat zo bv. de
mogelijkheid om lijsten van \emph{records} te defini\"eren.

In codevoorbeeld \ref{lst:foo-complex_type} wordt het \ttt{nodes}-domein
uitgebreid met een eigenschap, genaamd \ttt{inbox}. Deze bestaat uit meerdere
\emph{tuples} die op hun beurt bestaan uit een \ttt{timestamp} en meerdere
\ttt{bytes}.

\begin{listing}[ht]
  \begin{minted}[linenos,frame=lines,framesep=2mm,fontsize=\footnotesize]{javascript}
extend nodes with {
  inbox : [timestamp, byte*]* = []
}
  \end{minted}
  \vspace{-5mm}
  \caption{Voorbeeld van een complex type}
  \label{lst:foo-complex_type}
\end{listing}

%!TEX root=masterproef.tex

\section{Code generator}
\label{section:devel-codegen}

Met FOO-lang zijn we nu in staat om inbraaddetectiealgoritmes te beschrijven.
Deze FOO-lang code moet vervolgens door de code generator omgezet worden in een
gewone programmeertaal. In het geval van dit prototype is dat C.

%!TEX root=masterproef.tex

\subsection{Opbouw}

Figuur \ref{fig:devel-component-overview} geeft een overzicht van de opbouw van
de oplossing.

\begin{figure}[ht]
  \centering
  \includegraphics[width=\linewidth]{resources/component-overview.pdf}
  \caption{Overzicht van componenten en kernentiteiten}
  \label{fig:devel-component-overview}
\end{figure}

Intern bestaat de hele oplossing uit twee grote delen: het semantische en het
code gedeelte. Binnen het semantische gedeelte vinden we het SM terug. Dit
model kan benaderd worden door middel van een zgn. \emph{visitor}, een
implementatie van het \emph{visitor pattern}. Aan de hand van deze
\emph{visitor} kunnen transformaties van het model gerealiseerd worden.

Het SM is de primaire invoer voor de generator. Die kan zijn werk slechts
vervullen door middel van een compositie met \emph{platform-} en
\emph{domeininformatie}, een vertaler (Engels: \emph{Translator}) die elementen
uit het semantische gedeelte kan omzetten naar overeenkomstige elementen in het
code gedeelte en de uiteindelijk beoogde programmeertaal (Engels:
\emph{Language}).

De programmeertaal maakt deel uit van het CM. Dit is op zijn beurt opgebouwd
uit een hi\"erarchie van vier niveaus. De structuur van de beoogde code wordt
weergegeven door de compilatie \emph{unit}, de \emph{modules} en de
\emph{secties}, waarbij de unit staat voor het geheel, de modules voor
functioneel samenhangende delen en de secties zorgen voor een fysieke opdeling
in bv. bestanden. De juiste realisatie van deze hi\"erarchie wordt overgelaten
aan de implementatie van de taal die hier betekenis kan aan geven.

Op het laagste niveau van het CM vinden we de \emph{instructies}. Deze kunnen
gebruikt worden om de effectieve code voor te stellen. Er bestaat in het CM per
definitie een overeenkomstige instructie voor elk element uit het SM. Aangezien
het SM functioneel rijker is dan de meeste programmeertalen, zal na constructie
van het initi\"ele CM, door middel van transformaties, alternatieven
ge\"implementeerd moeten worden binnen de mogelijkheden van de uiteindelijke
programmeertaal.

\subsection{ANTLR}
\label{subsection:devel-antlr}

Maar alles begint bij het inladen van de FOO-lang bronbestanden in het SM. Dit
gebeurt door middel van een \emph{parser} die de tekstuele voorstelling
analyseert en de taal-eigen constructies er uit puurt. Het resultaat van deze
stap is de constructie van een boomstructuur die de juiste semantische
betekenis van de verschillende constructie structureel weergeeft. Zo'n
boomstructuur is een AST. Figuur \ref{fig:devel-ast} toont de AST van het
elementaire voorbeeld uit codevoorbeeld \ref{lst:hello.foo}.

\begin{figure}[ht]
  \centering
  \includegraphics[width=\linewidth]{resources/hello_ast.pdf}
  \caption{De AST van het elementaire voorbeeld, \ttt{hello.foo}}
  \label{fig:devel-ast}
\end{figure}

We herkennen duidelijk de inhoud van het codevoorbeeld: op het hoogste niveau
zien we de module met een naam, de definitie van een constante, een uitbreiding
van het domein, een functie definitie en een geannoteerde applicatie van een
functie op een domein. De AST is ontdaan van alle ondersteunende syntax zoals
aanduidingen voor blokken code \dots en bevat louter de semantische inhoud.

%!TEX root=masterproef.tex

\subsection{Interfaces}
\label{subsection:devel-codegen-interfaces}

Voor we het SM en het CM in detail bekijken, kijken we eerst naar de interfaces
die de code generator ter beschikking stelt.

\subsubsection{foo.py}

Op het hoogste niveau biedt de generator een commandolijn interface (CLI) aan
in de vorm van een Python script: \ttt{foo.py}. Codevoorbeeld \ref{lst:foo.py-help}
toont de uitvoering van het script die een overzicht geeft van de mogelijkheden.

\begin{listing}[ht]
  \begin{minted}[linenos,frame=lines,framesep=2mm,fontsize=\footnotesize]{console}
$ source setpath.sh
$ ./foo.py --help
usage: foo.py [-h] [-v] [-c] [-i] [-g FORMAT] [-o OUTPUT] [-l LANGUAGE]
              [-p PLATFORM]
              [sources [sources ...]]

Command-line tool to interact with foo-lang and its code generation
facilities.

positional arguments:
  sources               the source files in foo-lang

optional arguments:
  -h, --help            show this help message and exit
  -v, --verbose         output info on what's happening
  -c, --check           perform model checking
  -i, --infer           perform model type inferring
  -g FORMAT, --generate FORMAT
                        output format (choices: none, ast, ast-dot, sm-dot,
                        foo, code / default: none)
  -o OUTPUT, --output OUTPUT
                        output directory (default: .)
  -l LANGUAGE, --language LANGUAGE
                        when format=code: target language (choices: c /
                        default: c)
  -p PLATFORM, --platform PLATFORM
                        when format=code: target platform (choices: moose,
                        demo / default: moose)
  \end{minted}
  \vspace{-5mm}
  \caption{Informatie over de werking van \ttt{foo.py}}
  \label{lst:foo.py-help}
\end{listing}

De CLI biedt toegang tot alle aspecten van de generator: model controle
(\ttt{check}), type deductie (\ttt{infer}), het uitvoerformaat, waar de uitvoer
moet geplaatst worden, welke taal gebruikt moet worden en voor welk platform de
generatie moet gebeuren.

De lijst van mogelijke uitvoerformaten bestaat uit: \ttt{none}, \ttt{ast},
\ttt{ast-dot}, \ttt{sm-dot}, \ttt{foo} en \ttt{code}.

Formaat \ttt{ast} toont een hierarchisch overzicht van de AST op het scherm in
tekstuele vorm, zoals weergegeven in codevoorbeeld \ref{lst:foo.py-ast}. De
uitvoer van \ttt{ast-dot} zagen we in essentie reeds eerder in figuur
\ref{fig:devel-ast}. De uitvoer is feitelijk code die als invoer kan dienen
voor GraphViz \citep{url:graphviz}, een open bron project dat zich
specialiseert in het visualiseren van graafgeori\"enteerde gegevens, zoals deze
AST met een boomstructuur. Door middel van het \ttt{dot} commando kan
vervolgens van deze code een visuele voorstelling gemaakt worden.

\begin{listing}[ht]
  \begin{minted}[linenos,frame=lines,framesep=2mm,fontsize=\footnotesize]{console}
$ source setpath.sh
$ ./foo.py -g ast examples/hello.foo 
ROOT
  MODULE
    IDENTIFIER
      hello
    CONST
      IDENTIFIER
        interval
      UNKNOWN_TYPE
      INTEGER_LITERAL
        1000
    EXTEND
...
  \end{minted}
  \vspace{-5mm}
  \caption{Tekstuele uitvoer van een AST}
  \label{lst:foo.py-ast}
\end{listing}

Overeenkomstig bestaat er ook de mogelijkheid om een visuele voorstelling te
maken van het SM, door middel van het \ttt{sm-dot} formaat. Om controles te
doen betreffende de goede verwerking van de FOO-lang broncode kan een ingelezen
set van modules ook opnieuw als FOO-code uitgevoerd worden.

Tot slot is er nog het \ttt{code} formaat, dat de generator vraagt om
effectieve code te genereren. Hierbij dienen dan ook de overige opties
eventueel ingevuld te worden: uitvoerlocatie, taal en platform.

\subsubsection{API}

Het \ttt{foo.py} Python script is slechts een CLI-verpakking rond de Python
API. Deze biedt alle functionaliteit aan in de vorm van een Python module met
een imperatieve interface. Codevoorbeeld \ref{lst:codegen-api} toont de interface
van deze module.

\begin{listing}[ht]
  \begin{minted}[linenos,frame=lines,framesep=2mm,fontsize=\footnotesize]{python}
def create_model():
  ...
  return model

def parse(string, noprint=False):
  ...
  return parser

def infer(model, silent=False):
  ...

def check(model, silent=False):
  ...

def generate(model, args):
  ...

def load(string, model=None):
  ...
  return model
  \end{minted}
  \vspace{-5mm}
  \caption{API van de code generator}
  \label{lst:codegen-api}
\end{listing}

In volgorde zien we de verschillende fasen uit het generatie proces: het
aanmaken van een (leeg) model, het parsen van de broncode, het deduceren van
onbekende types, het controleren of een model volledig in orde is en
uiteindelijk het genereren van de code. De bijkomende \ttt{load} functie
combineert de \ttt{create\_model} en \ttt{parse} functionaliteit in \'e\'en
handige functie.

De API laat toe om de generator vanuit Python aan te spreken en eventueel
verder te integreren in een uitgebreider compilatieproces, of om andere
interfaces te voorzien (visuele gebruikersinterfaces zoals bv. een
webinterface \dots).

De API biedt toegang tot de entiteiten op het hoogste niveau, zoals de parser,
de model-entiteit uit het SM \dots De volledige openheid van Python code laat
verder toe om dieper door te dringen en elk aspect van het bv. model te
ondervragen of zelfs te wijzigen.

Beide onderliggende modellen zijn echter volledig ondervraagbaar aan de hand
van een \emph{visitor}. Deze worden door de generator veelvuldig gebruikt,
zelfs voor kleine operaties en bieden een veel aantrekkelijkere interface om
met de modellen te werken dan het ruwweg volgen van eigenschappen en methoden
doorheen het model.

\include{sematic-model}
%!TEX root=masterproef.tex

\subsection{Code model}
\label{subsection:devel-code-model}

Het CM staat in feite volledig los van de generator en het SM. Het is dan ook
als een op zich staand project ontwikkeld, als generieke beschrijving van
uitvoerbare code, genaamd \emph{CodeCanvas}.

\subsubsection{CodeCanvas}

CodeCanvas biedt een API om hi\"erarchische structuren te bouwen. Deze kunnen
opgebouwd worden uit zelf te defini\"eren entiteiten. Standaard voorziet
CodeCanvas de concepten \emph{unit}, \emph{module}, \emph{sectie} en
\emph{code}.

\begin{description}[noitemsep, topsep=1pt, partopsep=1pt]

  \item[Unit] is het hoogste niveau en verzamelt alle onderliggende
  \emph{modules}.
  
  \item[Module] komt overeen met een functioneel geheel en bestaat uit
  \emph{secties}.
  
  \item[Sectie] wordt functioneel ingevuld met \emph{code}instructies. Er
  worden standaard twee secties binnen een module aangemaakt: \'e\'en voor
  declaraties en \'e\'en voor definities.
  
  \item[Code] is een basisbouwsteen voor instructies. Standaard kan hier een
  tekstuele voorstelling aan gegeven worden. Praktisch zal men overerven van
  deze klasse en er een taxonomie mee opbouwen.

\end{description}

\noindent Daarnaast biedt CodeCanvas de mogelijkheid om entiteiten te markeren
met een label (\emph{tag}) en onderliggende entiteiten te selecteren of zoeken
op basis van die labels.

Dankzij een \emph{vloeiende} (\emph{fluent}) interface laat CodeCanvas
toe om zeer leesbare operaties te formuleren op deze hi\"erarchische
codestructuren.

Codevoorbeeld \ref{lst:codecanvas-hello} toont een eenvoudig voorbeeld dat de
meeste functionaliteit van CodeCanvas toepast. 

\begin{listing}[ht]
  \begin{minted}[linenos,frame=lines,framesep=2mm,fontsize=\footnotesize]{python}
from structure import Unit, Module
import instructions as code
import languages.C  as C

unit = Unit().append( Module("hello") )
main = unit.select("hello", "dec").append(code.Function(name="main"))
main.append(code.Print("Hello World\n"))

print str(unit)
print C.Emitter().emit(unit)
print str(unit)
  \end{minted}
  \vspace{-5mm}
  \caption{Werking van het \emph{CodeCanvas}}
  \label{lst:codecanvas-hello}
\end{listing}

Het voorbeeld construeert op regel 5 een \emph{unit} en voegt er een
\emph{module}, genaamd \ttt{hello}, aan toe. Achterliggend worden bij de
aanmaak van een module onmiddellijk 2 \emph{secties} toegevoegd, genaamd
\ttt{dec} en \ttt{def}.

Op regel 6 wordt vertrekkende van de \emph{unit} de \ttt{dec} \emph{sectie}
geselecteerd. De \ttt{select} methode laat toe om een opeenvolgende reeks van
\emph{labels} te defini\"eren die het pad vormen vanaf de startentiteit tot de
te selecteren entiteit.

Een gelijkaardige methode, \ttt{find}, accepteert een variabele lijst
argumenten en zoekt vervolgens, vertrekkende van de startentiteit, naar
entiteiten die alle opgegeven \emph{labels} dragen.

Beide methodes kunnen lijsten van entiteiten teruggeven. Op deze lijsten kunnen
eveneens alle methoden opgeroepen worden zoals op een enkele entiteit. Dit
resulteert in een zeer transparante interface.

De geselecteerde sectie wordt vervolgens een functie toegevoegd met de naam
\ttt{main}. Op regel 7 wordt aan deze functie een \ttt{print} opdracht
toegevoegd. Tot slot wordt de \emph{unit} op twee manieren uitgevoerd: eerst
door er een tekstuele voorstelling van te maken en in tweede instantie door
gebruik te maken van een programmeertaal, in dit geval C.

De uitvoer van dit programma is weergegeven in \ref{lst:codecanvas-output} en
toont eerst de technische tekstuele voorstelling van de hi\"erarchie. Tussen
vierkante haakjes staan de \emph{labels} die aan een entiteit verbonden zijn.
Effectieve instructie-implementaties tonen hun parameters als een verzameling
van de naam en de waarde.

\begin{listing}[ht]
  \begin{minted}[linenos,frame=lines,framesep=2mm,fontsize=\footnotesize]{console}
  Module hello [hello]
    Section def [def]
    Section dec [dec]
      Function {'params': (), 'type': void, 'id': main}
        Print {'args': (), 'string': "Hello World\n"}

  void main(void);
  #import <stdio.h>
  void main(void) {
    printf("Hello World\n");
  }
  
  Module hello [hello]
    Section def [def]
      Prototype {'params': [], 'type': void, 'id': main}
    Section dec [dec]
      Import {'imported': '<stdio.h>'} [import_stdio] <sticky>
      Function {'params': [], 'type': void, 'id': main}
        Print {'args': (), 'string': "Hello World\n"}
  \end{minted}
  \vspace{-5mm}
  \caption{Uitvoer van voorbeeld werking van het \emph{CodeCanvas}}
  \label{lst:codecanvas-output}
\end{listing}

Het tweede deel van de uitvoer toont de overeenkomstige C-code voor deze
hi\"erarchie. We merken op dat op regel 7 een prototype en op regel 8 een
\emph{import} opdracht verschijnen die niet in de hi\"erarchie voorkwamen.

Wanneer we een tweede maal de \emph{unit} omvormen tot een tekstuele
voorstelling, zien we deze twee entiteiten wel opduiken. De uitvoermodule voor
de programmeertaal C werkt in meerdere stappen.

Tijdens een eerste fase wordt de hi\"erarchie doorlopen en worden uitbreidingen
gedaan. In dit voorbeeld gebeuren er twee: wanneer een \emph{print} opdracht
gevonden wordt, wordt een \emph{import} opdracht toegevoegd die de declaraties
van \ttt{stdio.h} zal inladen. Een tweede transformatie zal voor elke
\emph{functie} een prototype aanmaken in de declaratiesectie van de module.

Deze fase zorgt ook voor het omzetten van constructies die niet standaard
ondersteund worden naar uitwerkingen met constructies die wel bestaan in de
beoogde taal. Een voorbeeld hiervan zijn bv. \emph{tuples}. Deze worden door de
transformatie naar C herschreven door middel van structuren en functies om deze
structuren te benaderen en onderhouden.

In een tweede fase doorloopt de uitvoermodule opnieuw de volledige
hi\"erarchie, maar vormt nu elke entiteit om in de overeenkomstige tekstuele C
syntax.

Beide fasen worden ge\"implementeerd door middel van \emph{visitors}. Deze
\emph{visitors} zijn tevens beschikbaar van buitenaf en laten toe om andere
transformaties te implementeren.

\subsubsection{Filosofie}

De doelstelling van CodeCanvas is het aanbieden van een API die toelaat om te
werken zoals een programmeur denkt/werkt tijdens programmeren, maar op basis
van een abstracte programmeertaal die een superset aanbiedt van constructies
uit verschillende programmeertalen.

Enkele typische eigenschappen die deze doelstelling ondersteunen zijn:

\begin{description}

  \item[Functionele cross-referenties] Door middel van \emph{labels} en de
  \emph{selectie}- of \emph{zoek}-functionaliteit kan er op functionele wijze
  omgegaan worden met code. Zo kan een gebruiker aan de beschrijving van een
  functie een \emph{label} toevoegen en hier later eenvoudig naar verwijzen om
  nog bijkomende logica toe te voegen.

  \item[Zoek-en-wijzig] Dankzij de functionele cross-referenties en de
  transparantie van een entiteit of een lijst van entiteiten is het eenvoudig
  om algemene aanpassingen door te voeren. Dit kan bv. gebruikt worden om aan
  het begin van alle declaratie-secties een standaard blok commentaar te
  plaatsen of om systematisch aanpassingen met betrekking tot naamgeving door
  te voeren.

  \item[Automatisch vervolledigen] Het voorbeeld van het automatisch toevoegen
  van \ttt{import} opdrachten of \ttt{prototypes} is een krachtige manier om de
  gebruiker te ontslaan van redundant en repetitief werk.

\end{description}

%!TEX root=masterproef.tex

\subsection{Transformaties}
\label{subsection:devel-transformations}

Uit de gedetailleerde voorgaande bespreking van het SM en het CM leren we dat
de kracht van de oplossing niet zozeer zit in de statische taxonomie die beide
modellen aanbieden, maar wel in de transformaties.

Het idee hiervoor werd ontleend aan een analyse- en ontwikkel-paradigma dat
gebruik maakt van zgn. model-gedreven architectuur (\emph{Model Driven
Architecture}) (MDA) \citep{soley2000model,kleppe2003mda}. MDA focust op UML,
maar de principes zijn overdraagbaar naar anderen modelvoorstellingen.

Het principe vertrekt van een hoog-niveau en zeer abstracte beschrijving van
een probleemdomein in de vorm van een platformonafhankelijk model
(\emph{Platform Independent Model}) (PIM) en evolueert door een opeenvolging
van (model) transformaties (MT) stapsgewijs tot een platform-specifiek model
(\emph{Platform Specific Model}) (PSM).

In bijlage \ref{appendix:visitor} wordt het \emph{visitor}-patroon voorgesteld.
Dit is de basis voor de implementatie van transformaties in de generator. Naast
het theoretische patroon wordt ook ingegaan op de technische uitwerking in
Python.


%!TEX root=masterproef.tex

\section{FOO-lib}
\label{section:devel-foo-lib}

Een laatste aspect van de implementatie bestaat uit de softwarebibliotheek die
de generatie vervolledigt. Deze bibliotheek biedt typisch twee types van
ondersteunende modules aan:

\begin{description}

\item[Algemene modules] - Dit kan vergeleken worden met de standaard
bibliotheek die tegenwoordig bij veel programmeertalen komt en veelal de kracht
er van bepaald. In het geval van deze eerste implementatie betreft het hier de
\ttt{crypto} module met functies om cryptografische sleutels te bepalen. Dit is
functionaliteit die te verwachten is bij implementatie van detectiealgoritmes.
De \ttt{time} module biedt toegang een klok en zorgt hiermee voor een
abstractielaag naar het onderliggende systeem.

\item[Knoopdomein-geori\"enteerde modules] - Het uitgangspunt van deze
masterproef is dat het mogelijk is om gemeenschappelijke logica betreffende
sensorknopen te centraliseren, om ze een meer optimale werking te bekomen. Het
is dan ook niet verwonderlijke dat FOO-lib de plaats is waar we deze
functionaliteit terugvinden. Het betreft hier de gemeenschappelijke parser voor
binnenkomende berichten, de planningsmodule voor planbare functieoproepen \dots

\end{description}

Vandaag dient er voor elk platform en programmeertaal een implementatie van
FOO-lib te gebeuren. Echter, indien deze code verder zou omgezet worden in een
vorm die rechtstreeks bruikbaar is voor de generator, zou deze logica mee
verwerkt kunnen worden door de generator, zou bv. \emph{inlining} kunnen
toegepast worden en zou nog verdere optimalisatie kunnen bereikt worden.

Misschien nog belangrijk is dat het onderhouden van verschillende
implementaties van FOO-lib sterk beperkt zou worden tot een nog lager niveau
van ondersteuning van de platformen en de talen.

%!TEX root=masterproef.tex

\section{Generatie}
\label{section:generation}

Alle voorgaande structurele componenten zorgen samen voor het generatieproces.
Dit proces accepteert FOO-lang broncode, importeert die in het SM,
transformeert die in een CM en uiteindelijk wordt dat CM uitgevoerd als
effectieve programmacode.

Bijlage \ref{appendix:hello-srcs} bevat de belangrijkste gegenereerde code voor
het elementaire voorbeeld, \ttt{hello.foo}. We overlopen enkele aspecten van
deze laatste stap in het generatieproces en tonen hoe de verschillende
componenten bijdragen tot bepaalde van de patroonmatige constructies.

\subsection{main.c}

Deze platform-implementatie gaat uit van de eenvoudigste opzet, op basis van
een expliciete \emph{event loop} (regels 18 tot 24). Verder voorziet de
generatie aan de hand van commentaar enkele instructies voor de gebruiker om
applicatie-specifieke code toe te voegen of om het gebruikte onderliggende
raamwerk te initialiseren (zie respectievelijk regels 9 en 4 in de broncode van
\ttt{main.c}).

We merken vooral de functie-oproepen op naar functies met een
``\ttt{nodes\_}''-prefix. Deze gaan naar de \ttt{nodes} knoopgeori\"enteerde
module die deel uitmaakt van de FOO-lib softwarebibliotheek. Op regel 28 zien
we bv. de registratie van de uitvoerstrategie die na het verstrijken van een
intervaltijd telkens de functie \ttt{step} zal oproepen.

Die \ttt{interval} constante is gedefinieerd in het \ttt{constants.h} bestand,
samen met eventueel andere constanten. De generator tracht om functioneel
verwante zaken bij elkaar te plaatsen. Zo gaan dus alle constanten in \'e\'en
bestand, maar ook het importeren van functionaliteit wordt gecentraliseerd om
de te genereren code eenvoudiger te maken. Zo worden bv. alle nodig
\emph{\#include} opdrachten ook in \'e\'en \ttt{includes.h} bestand
samengebracht.

\subsection{node\_t.h en node\_t.c}

H\'et centrale gegeven is natuurlijk de voorstelling van een sensorknoop. De
generator zal de informatie uit verschillende FOO-lang modules trachten samen
te brengen tot \'e\'en gemeenschappelijk voorstelling.

Standaard heeft een knoop twee eigenschappen, nodig voor de interne werking van
de \ttt{nodes} module: een unieke, interne identificatie (\ttt{id}) en het
netwerkadres van de knoop (\ttt{address}). Daarnaast voegt de generator alle
uitgebreide eigenschappen toe en maakt van het geheel een structuur en een
\ttt{node\_t} type.

Naast de declaratie van het type, wordt ook een functie voorzien om een nieuwe
knoop te initialiseren. Deze functie bevat opdrachten die overeenkomen met de
gedefinieerde initi\"ele waarden van de eigenschappen.

\subsection{nodes-\emph{module}.h en nodes-\emph{module}.c}

Tot slot zal de generator het \ttt{nodes} domein de kans bieden om voor elke
FOO-lang module een CM module te maken. Hierin wordt de functionaliteit die
eigen is aan de module ondergebracht. Typisch vinden we hier de functies terug
die eerder gekoppeld werden aan een uitvoerstrategie.

In het geval van het voorbeeld is dit de \ttt{step} functie.

\subsection{Communicatie}

Aan het elementaire voorbeeld ontbreekt natuurlijk \'e\'en heel belangrijk
aspect: communicatie. Er worden geen berichten verstuurd, noch ontvangen. De
generator zal het versturen van berichten delegeren naar de \ttt{nodes} FOO-lib
module en zal voor het verwerken van ontvangen berichten functies registreren
bij diezelfde module. Deze worden vervolgens opgeroepen indien een binnenkomend
bericht voldoet aan de eisen voor die specifieke verwerkingsfunctie.

Codevoorbeelden \ref{lst:comm.foo} en \ref{lst:comm.c} illustreren het typische
communicatiepatroon en de overeenkomstige generatie.

\begin{listing}[ht]
  \begin{minted}[linenos,frame=lines,framesep=2mm,fontsize=\footnotesize]{javascript}
after nodes receive do function(me, sender, from, hop, to, payload) {
  // payload is a list of data. we can consider one or more cases
  case payload {
    // e.g. we can check if we find an atom and three variables after is
    contains( [ #heartbeat, time:timestamp, sequence, signature:byte[20] ] ) {
      ...
    }
  }
}
  \end{minted}
  \vspace{-5mm}
  \caption{Verwerking van een binnenkomend bericht in FOO-lang}
  \label{lst:comm.foo}
\end{listing}

Via een uitvoerstrategie wordt een (anonieme) functie gekoppeld aan het
ontvangen van een bericht door een knoop. Vervolgens wordt gecontroleerd op
verschillende mogelijke patronen. Een bericht dat een \ttt{\#heartbeat}
\emph{atoom} bevat, zal de volgende bytes koppelen aan drie variabelen:
\ttt{time}, \ttt{sequence} en \ttt{signature}.

\begin{listing}[ht]
  \begin{minted}[linenos,frame=lines,framesep=2mm,fontsize=\footnotesize]{c}
void init(void) {
  ...
  payload_parser_register(nodes_process_incoming_case_0, 2, 0x00, 0x01);
  ...
}
...
void nodes_process_incoming_case_0(node_t* me, node_t* sender, node_t* from,
                                   node_t* hop, node_t* to, payload_t* payload) {
  // extract variables from payload
  uint32_t time = payload_parser_consume_timestamp();
  uint8_t sequence = payload_parser_consume_byte();
  uint8_t* signature = payload_parser_consume_bytes(20);
  ...
}
  \end{minted}
  \vspace{-5mm}
  \caption{Gegenereerde code voor een binnenkomend bericht}
  \label{lst:comm.c}
\end{listing}

De generator zal de verwerking van een binnenkomend bericht detecteren en dit
uit elkaar halen en structureren aan de hand van een functie en een registratie
van die functie.

Het \emph{atoom} is hier omgezet in twee opeenvolgende bytes: \ttt{0x00, 0x01}.
Wanneer de algemene parser deze sequentie ontmoet, zal
\ttt{nodes\_process\_incoming\_case\_0} opgeroepen worden. De signatuur van
deze functie bevat alle informatie over het bericht ivm afzender,
bestemmeling\dots

Het vervolg van het gezochte patroon vinden we hier tevens terug. De drie
variabelen worden gedeclareerd en ge\"initialiseerd door middel van een aantal
functies die de volgende bytes uit het bericht zullen omzetten naar een
gewenste voorstelling.

\subsection{\emph{Tuples}, lijsten en meer patroonherkenning}

Net zoals het \ttt{node\_t} type, worden voor \emph{tuples} ook structuren
aangemaakt en functies gegenereerd die de behandeling van het \emph{tuple}-type
toelaten. Alle declaraties van de types en de definities van de manipulerende
functies worden samengebracht in respectievelijk \ttt{tuples.h} en
\ttt{tuples.c}. Codevoorbeeld \ref{lst:tuples.h} toont de typische module
interface van een gegenereerd \emph{tuple}.

\begin{listing}[ht]
  \begin{minted}[linenos,frame=lines,framesep=2mm,fontsize=\footnotesize]{c}
typedef struct tuple_0_t {
  uint32_t elem_0;
  payload_t* elem_1;
  struct tuple_0_t* next;
} tuple_0_t;
tuple_0_t* make_tuple_0_t(uint32_t elem_0, payload_t* elem_1);
void free_tuple_0_t(tuple_0_t* tuple);
tuple_0_t* copy_tuple_0_t(tuple_0_t* source);
  \end{minted}
  \vspace{-5mm}
  \caption{Gegenereerde code voor een \emph{tuple}}
  \label{lst:tuples.h}
\end{listing}

Aangezien een \emph{tuple} in FOO-lang gedefinieerd wordt op basis van types,
worden standaardnamen gebruikt voor de elementen. Tevens voorziet een
verwijzing naar zichzelf, om de constructie van lijsten toe te laten. Drie
functies bieden de mogelijkheid om een \emph{tuple} aan te maken, vrij te geven
of te kopi\"eren.

\emph{Tuples} worden meestal in combinatie met lijsten gebruikt. Codevoorbeeld
\ref{lst:lists.c} toont enkele fragmenten uit de generatie van functies om lijsten
te onderhouden.

\begin{listing}[ht]
  \begin{minted}[linenos,frame=lines,framesep=2mm,fontsize=\footnotesize]{c}
void list_of_tuple_0_ts_push(tuple_0_t** list, tuple_0_t* item) {
  item->next = *list;
  *list = item;
}
...
uint16_t list_of_tuple_0_ts_remove_match_lt_now(tuple_0_t** list) {
  ...
  while((iter != NULL)) {
    if((iter->elem_0 < now())) {
      ...
    }
  }
}
  \end{minted}
  \vspace{-5mm}
  \caption{Gegenereerde code voor manipulatie van lijsten}
  \label{lst:lists.c}
\end{listing}

De eerste functie voegt een instantie van het \emph{tuple}-type toe aan een
lijst van dat type. Hierbij wordt de eerder gedefinieerde verwijzing gebruikt.

Het tweede voorbeeld is iets complexer. Het betreft een functie om een element
uit diezelfde lijst te verwijderen. De conditie om zo'n element te verwijderen
zit ingebakken in de functie, zowel in de naam als in de logica. De oorsprong
hiervan kunnen we terugvinden in de overeenkomstige FOO-lang broncode:
\mint{javascript}| failures = node.queue.remove([ < now(), _ ]) |

Met deze opdracht wordt de eigenschap \ttt{queue} op een knoop object
geraadpleegd. Deze eigenschap is gedeclareerd als een lijst van het voorgaande
\emph{tuple}-type. Het argument van de \ttt{remove} methode, is een patroon dat
bestaat uit een conditie \ttt{< now()} en een ``\_'' die
syntactisch aanduidt dat alle waarden op deze positie in het \emph{tuple} goed
zijn.

De generator zal zo'n patroon analyseren en trachten om specifieke code te
genereren. Een alternatief zou zijn om generieke lijst-functies op te namen in
FOO-lib. Vervolgens zouden de condities, ge\"implementeerd als kleine
gegenereerde functies, meegegeven kunnen worden aan de generieke functies. Deze
kleine functies zouden dan opgeroepen kunnen worden om de effectieve condities
te testen.

In het prototype is hier gekozen om een voorbeeld te introduceren van
\emph{inlining}. Het gebruiken van functie verwijzingen zou het genereren wel
eenvoudiger maken, doch de kost die gepaard gaat met het herhaaldelijk oproepen
van de verwijzing naar de functie met de conditie zou een serieuze belasting
voor de \mcu met zich meebrengen. In het kader van dit probleem werden testen
gedaan die uitwezen dat het verschil tussen het gebruik van een verwijzing en
het \emph{inline}'en van deze code eenvoudig kan oplopen tot een factor 3000.

\vspace{5mm}

Het feit dat het dankzij codegeneratie die start op een functioneel niveau
mogelijk is om zulke optimalisaties door te voeren, illustreert de kracht en de
mogelijkheden die code generatie voor dit soort van problemen kan bieden.

De generatiepatronen die ge\"implementeerd zijn in dit prototype zijn niet
volledig en zeker niet geoptimaliseerd. Alleen hier al liggen grote
opportuniteiten tot verbetering. De doelstelling was om code te genereren die
vergelijkbaar was met overeenkomstige manuele code. In een volgend hoofdstuk
willen we immers beide implementaties vergelijken en zien of the theoretische
winst die de beter georganiseerde gegenereerde code zou moeten opleveren, zich
ook effectief in praktijk manifesteert.

