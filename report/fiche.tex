%!TEX root=masterproef.tex

\setup{
  title={Verlagen van de impact van inbraakdetectie
         in draadloze sensornetwerken
         door middel van een domeinspecifieke taal
         en codegeneratietechnieken},
  author={Christophe Van Ginneken},
  promotor={Prof.\,dr.\,ir.\ Wouter Joosen \and Prof.\,dr.\,ir.\ Christophe Huygens},
  assessor={Dr.\ Benjamin Negrevergne \and Dr.\ Nelson Matthys},
  assistant={Drs.\,ir.\ Jef Maerien}
}

\setup{
  filingcard,
  translatedtitle={Lowering the Impact of Intrusion Detection
                   in Wireless Sensor Networks
                   using a Domain Specific Language
                   \& Code Generation Techniques},
  udc=681.3,
  shortabstract={
Draadloze sensornetwerken treden met rasse schreden onze persoonlijke
levenssfeer binnen. Beveiliging tegen inbraken moet garanties bieden dat deze
vooruitgang zelf geen bedreiging wordt. Preventie is de eerste stap, maar niet
alle inbraken kunnen vermeden worden. Soms moeten we genoegen nemen met het
detecteren ervan om ons in de toekomst er beter tegen te wapenen. Het
introduceren van inbraakdetectie in draadloze sensornetwerken resulteert al
snel in een gevecht om middelen: een draadloze sensorknoop beschikt over een
beperkte autonomie en moet zijn energie optimaal benutten. Inbraakbeveiliging
vraagt veel van de beschikbare middelen en hypothekeert daarmee de kans om
opgenomen te worden in het uiteindelijke ontwerp van elke nieuwe draadloze
sensorknoop. Indien het probleem niet kan vermeden worden, moeten we trachten
het draaglijker te maken. De introductie van inbraakdetectie heeft een impact
op verschillende vlakken. Deze masterproef wil zowel de druk op de middelen van
de sensorknopen verlichten als de bijkomende economische druk op de
ontwikkeling reduceren. Om dit te realiseren, wordt een domeinspecifieke taal
voorgesteld die onderzoekers in staat stelt om algoritmen voor inbraakdetectie
op een formele en platformonafhankelijke manier te defini\"eren. Deze eerste
stap ontslaat ontwikkelaars van nieuwe sensornetwerken van de taak om
onderzoeksliteratuur te doorworstelen en algoritmen uit deze teksten te puren.
Een formele beschrijving laat verder toe om deze algoritmen op geautomatiseerde
wijze te benaderen. Zo wordt het mogelijk om door middel van codegeneratie de
algoritmen automatisch om te zetten in platformspecifieke programmacode, z\'o
georganiseerd dat de middelen van de sensorknoop zo optimaal mogelijk benut
worden. De initi\"ele testen met een prototype codegenerator zijn veelbelovend.
Ze bevestigen de intu\"itie dat een goede organisatie van verschillende
detectiealgoritmen kan leiden tot een beter gebruik van de middelen van een
sensorknoop \'en dat dit volledig geautomatiseerd kan gebeuren. Dankzij het
vrijwaren van de middelen van de sensorknoop en het reduceren van de
economische kost, wordt het zo mogelijk om meer inbraakpogingen te detecteren.
Zowel de domeinspecifieke taal als de codegenerator bieden opportuniteiten tot
verder onderzoek. Testen met detectiealgoritmen en platformen moeten op grotere
schaal uitgewerkt worden. De realisatie van een ecosysteem rond de
geformaliseerde detectiealgoritmen is een andere belangrijke richting die
nagestreefd moet worden en waar vooral het onderzoeksdomein baat bij heeft.
} }
