%!TEX root=masterproef.tex

\chapter{IDP en een co\"operatief algoritme}
\label{appendix:idp-cooperation}

\citep{krontiris2009cooperative} beschrijft enerzijds een theoretische model
voor het zgn. ``Intrusion Detection Problem'' (IDP) en biedt anderzijds een
zeer praktisch algoritme om gedistribueerd tot een consensus te komen voor het
ontmaskeren van een kwaadwillige knoop in het netwerk.

\section{IDP}
\label{section:idp}

Het theoretische model beschrijft het IDP aan de hand van $S = \{ s_1, s_2,
\dots s_n \}$, de set van sensoren in het netwerk, $N(s)$, de set van buren van
$s$ en $D(s)$, de set van knopen die door $s$ verdacht worden. Indien $|D(s)| =
1$, is de aanvaller ge\"identificeerd.

Enkele predicaten worden gedefinieerd als volgt: $source(q)$ geldt indien $q$
de aanvaller is, $honest(s) \iff \neg source(s)$, $expose_s(q) \iff D(s) = \{ q
\}$ ofwel $s$ verklaart dat $q$ de aanvaller is en $A(s) \iff D(s) \not= \{\}$
wat zoveel betekent als dat $s$ gealarmeerd is. De set van gealarmeerde buren
van een knoop $s$ wordt gedefinieerd als $AN(s) = \{ t | A(t) \wedge t \in N(s)
\}$ en de set van gealarmeerde buren die nuttig is voor een knoop $s$ wordt dan
uitgedrukt als $\tilde{AN}(q,s) = AN(q) \backslash \{s\}$.

Het IDP wordt vervolgens gedefinieerd als het vinden van een algoritme dat
voldoet aan de eigenschappen van correctheid: $\forall s \in S : honest(s)
\wedge expose_s(s') \implies A(s) \wedge source(s')$ en eindigheid: bij een
aanval zullen na een tijd alle eerlijke knopen in de gealarmeerde set een knoop
verdenken.

Twee condities worden voorgesteld: de ``Intrusion Detection Condition'' of IDC
en de ``Neighbourhood Conditions'' of NC. Indien aan minstens \'e\'en van deze
condities voldaan is, is het IDP oplosbaar.

De IDC wordt beschreven als $\forall p,q \in S : source(q) \implies
\tilde{AN}(p,q) \not= \tilde{AN}(q,p)$. Dit drukt uit dat geen enkele andere
knoop een zelfde gealarmeerde buurt kan hebben als de aanvaller.

De NC worden beschreven door ``\emph{alle buren van een aanvaller zijn
gealarmeerd}'' ($NC_1$) en ``\emph{indien twee of meer knopen verdacht zijn door
een meerderheid van knopen, hebben de eerlijke knopen niet-gealarmeerde buren}''
($NC_2$).

Figuur \ref{fig:idp-examples} illustreert het IDP en toepassing van IDC en NC
aan de hand van enkele voorbeelden:

\begin{figure}[ht]
\centering
\begin{subfigure}{.49\textwidth}
  \centering
  \includegraphics[width=.8\linewidth]{./resources/idp-nc-s1.pdf}
  \caption{}
  \label{fig:idp-examples-1}
\end{subfigure}
\begin{subfigure}{.49\textwidth}
  \centering
  \includegraphics[width=.8\linewidth]{./resources/idp-nc-s2.pdf}
  \caption{}
  \label{fig:idp-examples-2}
\end{subfigure}
\caption[Voorbeelden van de toepassing van IDC en NC]{Voorbeelden van de
toepassing van IDC en NC: Rode knopen zijn aanvallers, omcirkelde knopen zijn
gealarmeerd. $x \rightarrow y$ betekent dat knoop $x$ knoop y verdenkt}
\label{fig:idp-examples}
\end{figure}

De situatie in figuur \ref{fig:idp-examples-1} voldoet niet aan de IDC omdat
$\tilde{AN}(a,b) = \{c\} = \tilde{AN}(b,a)$. Maar in dit geval is wel voldaan
aan beide NC. Het IDP kan in dit geval opgelost worden aan de hand van een
deterministisch algoritme. Omdat er slechts \'e\'en knoop het hoogste aantal
verdenkingen op zijn naam heeft staan, kunnen knopen $b$ en $c$ eenvoudig
beslissen dat knoop $a$ de aanvaller is.

Figuur \ref{fig:idp-examples-2} toont een situatie waar de IDC wel voldaan is
want $\tilde{AN}(q,r) = \{s\} \not= \tilde{AN}(r,q) = \{\}$. Er zijn echter nu
twee knopen met een meerderheid aan verdenkingen: $a$ en $c$. Vanuit het
standpunt van knoop $b$ moet dus knoop $a$ of knoop $d$ valse aantijgingen
verspreiden. Als \'e\'en van deze knopen de aanvaller is, dan moet
$\tilde{AN}(a,d) \not= \tilde{AN}(d,a)$, anders is niet voldaan aan de IDC. Dit
impliceert tevens dat $\exists x : A(x) \wedge ( x \not\in N(a) \vee x \not\in
N(d) )$, ofwel dat er een knoop bestaat die gealarmeerd is, maar geen buur is
van de andere eerlijke knoop van de twee verdachte knopen. In dit geval zijn
knopen $b$ en $d$ inderdaad geen buren, maar beide verdenken knoop $a$.

\section{Algoritme}
\label{section:cooperation-algorithm}

Naast een theoretisch model, introduceren \citep{krontiris2009cooperative}
tevens een algoritme dat kan dienen als raamwerk voor co\"operatieve
inbraakdetectie. Het algoritme bestaat uit vier tot vijf fasen: initialisatie,
stemming, publicatie van gebruikte sleutels, ontmaskeren van de aanvaller en
optioneel het inroepen van informatie uit de externe kring. Het is in essentie
een implementatie van het Guy Fawkes protocol, beschreven in
\citep{anderson1998new}, dat toelaat om een reeks van berichten te authenticeren
op basis van \'e\'en initi\"eel gedeelde sleutel.

Tijdens de initialisatie fase krijgt elke knoop een unieke sleutel, $K_l$. Aan
de hand van deze sleutel wordt een \'e\'en-richtingsketting van langte $l$
gemaakt van sleutels door het toepassen van bv. SHA1 \citep{rfc:3174} hashing
toe te passen: $\{K_0, K_1, \dots K_{l-1}\}$ waarbij $\forall k \in [1..l] :
K_{k-1} = SHA1(K_k)$. Daarnaast worden ook naburige knopen gezocht tot twee
knopen ver en wordt sleutel $K_0$ gecommuniceerd aan al deze buren. De
volledige initialisatie fase wordt verondersteld te gebeuren zonder
mogelijkheid tot inbraken.

Tijdens de stemming brengen alle knopen een stem uit van de vorm $m_v(s),$ $
MAC_{K_j}(m_v(s))$. $m_v(s)$ bestaat uit een lijst van knopen die door knoop
$s$ beschouwd worden als mogelijke aanvallers, of uitgedrukt aan de hand van
het theoretische model: $m_v(s) = D(s)$. De $MAC_{K_j}()$ functie is een zgn.
\emph{Message Authentication Code} \citep{rfc:2104} en wordt meestal berekend
door het toepassen van een \'e\'en-richtings-hashfunctie toe te passen op de
boodschap. Typisch wordt er aan de boodschap een unieke, wederzijds gekende
identificatie toegevoegd, zodat de ontvanger van een boodschap deze
handtekening ook kan berekenen en vergelijken met het origineel. In dit geval
wordt $K_j$ gebruikt, waarbij $j$ de volgende indexwaarde krijg uit lijst van
beschikbare sleutels.

Een dergelijke boodschap kan op ogenblik van verzending door geen enkele andere
knoop gevalideerd worden. De enige sleutel die zij tot op dat ogenblik kennen
is de vorige en de vorige is net het resultaat van een SHA1 operatie op de
volgende. Dit betekent dat ook een mogelijke aanvaller niet in staat is om de
boodschap te wijzigen.

Pas wanneer de knopen elkaars stemmen hebben ontvangen, wordt de gebruikte
sleutel vrijgegeven in de publicatie fase. Op dit ogenblik kunnen de knopen de
eerder verstuurde stemmen valideren. Ze kunnen eerst controleren of de
gepubliceerde sleutel inderdaad de juiste is. Immers, door het toepassen van
SHA1 op deze sleutel, moeten zij de huidige gekende sleutel bekomen. Na
validatie van de nieuwe sleutel, kan ook de boodschap gevalideerd worden.

Nu alle knopen de stemmen van alle knopen ontvangen hebben en zeker zijn dat de
stemmen authentiek zijn, kan met \'e\'enzelfde algoritme door alle knopen de
aanvaller bepaald worden tijdens de ontmaskering fase.

Het is echter mogelijk dat er meerdere knopen zijn met eenzelfde aantal stemmen
zijn, iets dat typisch overeenkomt met een gelijke samenstelling van
gealarmeerde buren, wanneer de IDC niet kan gerealiseerd worden. Onder
voorbehoud dat aan de NC wel voldaan wordt, kan door het inroepen van de niet
gealarmeerde buren van de gealarmeerde knopen. Deze zullen op hun beurt nagaan
of de knopen die verdacht worden, buren zijn en dan aangeven dat zij hen
\emph{niet} verdenken. Aan de hand van deze informatie kunnen de eerder
gealarmeerde knopen hun beslissing verder staven.
