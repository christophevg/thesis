%!TEX root=masterproef.tex

\section{FOO-lang}
\label{section:devel-foo-lang}

De echt belangrijke taal in dit geval is niet Python, maar FOO-lang.
Codevoorbeeld \ref{lst:hello.foo} toont de implementatie van een elementair
voorbeeld in FOO-lang. Aan de hand van dit voorbeeld introduceren we de
typische bouwstenen van FOO-lang en doorlopen we het generatieproces.

\begin{listing}[ht]
  \inputminted[linenos,frame=lines,framesep=2mm,fontsize=\footnotesize]{js}{../src/foo-lang/examples/hello.foo}
  \vspace{-3mm}
  \caption{Elementair voorbeeld in FOO-lang: \ttt{hello.foo}}
  \label{lst:hello.foo}
\end{listing}

De code start op regel 6 met de declaratie van een \emph{module}. Een module is
een op zich staand geheel en zou bv. een detectiealgoritme kunnen zijn. Alles
wat volgt op de declaratie van de module, maakt er deel van uit.

Op regel 8 introduceren we een \emph{const}ante, \ttt{interval} en stellen die
gelijk aan \ttt{1000}. Hier zien we een eerste voorbeeld van het ontbreken van
expliciete typering in FOO-lang. Dankzij deductie van types zal in dit geval
het type van \ttt{interval} overeenkomen met een \emph{IntegerType}, omdat de
waarde \ttt{1000} gevormd is als een integer getal.

Ofschoon FOO-lang ge\"introduceerd wordt als DSL voor inbraakdetectie in WSN,
specificeert het zijn domein als dat van \emph{sensorknopen} of \emph{nodes}.
algoritmen met betrekking tot inbraakdetectie in WSN, hebben \'e\'en belangrijk
gemeenschappelijke entiteit en dat zijn de sensorknopen. Deze communiceren met
elkaar en op basis van die communicatie zijn zowat alle algoritmen opgebouwd.

Het raamwerk van de generator beheert het concept van een knoop of \emph{node}.
De algoritmen krijgen toegang tot deze knopen via een aantal functionele
constructies en ze kunnen de basisdefinitie van een knoop in het domein
uitbreiden met eigen eigenschappen. Dit gebeurt bv. op regel 10, waar (de knoop
van) het domein uitgebreid wordt met een eigenschap \ttt{sequence}. Deze
eigenschap wordt expliciet getypeerd als een \emph{byte} en krijgt als
initi\"ele waarde \ttt{0}.

FOO-lang is een \emph{functie}-geori\"enteerde taal die tracht om de functies
in de verschillende modules zo te organiseren dat de uitvoering ervan de \mcu
of de draadloze radio zo min mogelijk belast. Op regel 14 wordt een functie
gedefinieerd, genaamd \ttt{step}. Ze accepteert \'e\'en parameter, genaamd
\ttt{node}. We merken opnieuw op dat deze parameter niet getypeerd is.

De inhoud van de functie bestaat uit vertrouwde codeconstructies die bijna
gewone C-code kunnen zijn. Een conditie, een eigenschap, een
waardeverhoging\dots We zien hier onze eerder toegevoegde eigenschap,
\ttt{sequence}, terug opduiken.

Regel 20 brengt alle voorgaande definities samen in een
\emph{uitvoeringsstrategie}. FOO-lang tracht door middel van zijn syntax
leesbaar te zijn als een natuurlijke taal. Indien we regel 20 luidop lezen, kan
dit resulteren in: ``\emph{At every (passing of) interval with nodes do (the
function named) step.}''. En dat is exact wat deze regel definieert.

In deze ene regel zien we de lus over alle gekende knopen te voorschijn komen.
In alle algoritmen komt deze wel in \'e\'en of andere vorm terug. De lus is nu
echter geabstraheerd tot zijn functionele betekenis en onder controle van de
generator.

\subsection{Syntax en grammatica}
\label{subsection:devel-foo-lang-grammar}

Het voorgaande voorbeeld gebruikt slechts een kleine subset van de volledige
mogelijkheden van FOO-lang. De volledige grammatica van FOO-lang, zoals
gedefinieerd in het kader van dit prototype, is opgenomen in bijlage
\ref{appendix:foo-lang-grammar}. Deze bevat de \emph{Extended Backus-Naur Form}
(EBNF) die de taal eenduidig bepaalt. We bespreken hier kort NOG enkele andere
constructies.

\vspace{-3mm}

\subsubsection{Importeren van functionaliteit}

Het is mogelijk om externe functies te importeren. Dit gebeurt aan de hand van
de constructie \ttt{from ... import ... }, die een functie importeert vanuit
een module. Zo'n module kan een andere FOO-lang module zijn, of een extern
gedefinieerde functie uit een softwarebibliotheek. In
\ref{section:devel-foo-lib} introduceren we de FOO-lib, de standaard
softwarebibliotheek die de codegeneratie vervolledigt. Deze bevat tal van
functionaliteit die voorkomt in beschrijvingen van detectiealgoritmen.

\vspace{-3mm}

\subsubsection{Reageren op gebeurtenissen}

Naast het herhaaldelijk uitvoeren van een functie is het ook mogelijk om een
functie te koppelen aan een gebeurtenis in de context van het domein en de
knopen. In plaats van de \ttt{with ... do} constructie is het mogelijk om
v\'o\'or of n\'a (\ttt{before} en \ttt{after}) de uitvoering van een functie
een andere functie uit te voeren. Codevoorbeeld \ref{lst:foo-event_handler}
geeft een eenvoudig voorbeeld waarbij we ontvangen berichten tellen als deze
aan ons geadresseerd waren.

\begin{listing}[ht]
  \begin{minted}[linenos,frame=lines,framesep=2mm,fontsize=\footnotesize]{javascript}
after nodes receive do function(me, sender, from, hop, to, payload) {
  if(to == me) {
    me.msg_count++
  }
}
  \end{minted}
  \vspace{-5mm}
  \caption{Voorbeeld van het reageren op een gebeurtenis}
  \label{lst:foo-event_handler}
\end{listing}

In dit voorbeeld merken we ook op dat functies anoniem kunnen gedefinieerd
worden en niet louter eerst met een naam.

\vspace{-3mm}

\subsubsection{Verschillende situaties afhandelen}

Het \ttt{case statement} laat toe om eenzelfde expressie te evalueren in
verschillende situaties. Typisch gebruik voor deze constructie is het
analyseren van ontvangen gegevens. Codevoorbeeld \ref{lst:foo-case} illustreert dit.

\begin{listing}[ht]
  \begin{minted}[linenos,frame=lines,framesep=2mm,fontsize=\footnotesize]{javascript}
after nodes receive do function(me, sender, from, hop, to, payload) {
  case payload {
    contains([#marker1, value]) {
      sender.msg_sent1++
    }
    contains([#marker2, value]) {
      sender.msg_sent2++
    }
    else {
      sender.msg_other++
    }
  }
}
  \end{minted}
  \vspace{-5mm}
  \caption{Voorbeeld van het afhandelen van verschillende situaties}
  \label{lst:foo-case}
\end{listing}

In dit voorbeeld worden de ontvangen berichten geanalyseerd en gecatalogeerd op
basis van de inhoud. Als in een bericht \ttt{\#marker1} wordt gevonden wordt
\ttt{msg\_sent1} verhoogd, in het geval van \ttt{\#marker2}, \ttt{msg\_sent2}
en anders \ttt{msg\_other}.

Dit voorbeeld introduceert tevens nog drie andere belangrijke constructies: het
\emph{atom}, lijsten en patroonkoppeling.

\vspace{-3mm}

\subsubsection{Atomen}

\emph{Atomen} werden ontleend aan Erlang \citep{armstrong1993concurrent}. Ze
stellen een uniek herkenbare entiteit voor. Het is aan de generator om met hulp
van het platform en/of domein en/of doeltaal hiervoor een geschikte
voorstelling te vinden.

In het geval van dit prototype werden twee bytes voorzien om een unieke
identificatie te maken van delen in berichten. De generator kan verschillende
strategie\"en volgen en beslissen op basis van de verschillende atomen die hij
tegenkomt.

\vspace{-3mm}

\subsubsection{Lijsten}

Lijsten worden voor verschillende doeleinden gebruikt. Ze worden syntactisch
gespecificeerd door middel van vierkante haken. Zo worden ze meestal als een
letterlijke voorstelling van de lijst opgenomen. In het voorbeeld in listing
\ref{lst:foo-case} is de \ttt{payload} parameter zo'n lijst. Ook het argument
van de \ttt{contains} methode die toegepast wordt op \ttt{payload} is een
lijst. De parameter verwacht echter niet echt een lijst, maar een patroon. In
dit geval is de lijst een deel van het patroon.

\vspace{-3mm}

\subsubsection{Patroon koppeling}

Door middel van patronen kan er een kopping gemaakt worden tussen gegevens. In
het voorbeeld in codevoorbeeld \ref{lst:foo-case} accepteert de
\ttt{contains}-methode op een \emph{lijst} een patroon. Dit patroon is een
lijst en bestaat uit variabele en niet-variabele elementen. De
\ttt{contains}-methode zal trachten de niet-variabele elementen te herkennen in
de lijst en vervolgens bij succesvolle herkenning een koppeling te maken tussen
de variabele elementen en de waarden in de oorspronkelijke lijst.

\vspace{-3mm}

\subsubsection{Complexe types}

In codevoorbeeld \ref{lst:hello.foo} zagen we reeds kort een voorbeeld van
typering. De \ttt{sequence} eigenschap werd getypeerd als een \ttt{byte}. Naast
\ttt{byte} bestaan ook \ttt{integer}, \ttt{float}, \ttt{boolean} en
\ttt{timestamp} als standaard eenvoudige types.

Vergelijkbaar met lijsten is er ook het \emph{tuple} type. Dit is een lijst van
types en defini\"eert de types van de elementen van een lijst van vaste lengte.

Van alle types kan ook een veelvoud gedefinieerd worden door het toevoegen van
een sterretje (\ttt{*}) achter het type. Zo kunnen lijsten van een bepaald type
gedefinieerd worden. Gecombineerd met het \emph{tuple} type, ontstaat zo bv. de
mogelijkheid om lijsten van \emph{records} te defini\"eren.

In codevoorbeeld \ref{lst:foo-complex_type} wordt het \ttt{nodes}-domein
uitgebreid met een eigenschap, genaamd \ttt{inbox}. Deze bestaat uit meerdere
\emph{tuples} die op hun beurt bestaan uit een \ttt{timestamp} en meerdere
\ttt{bytes}.

\begin{listing}[ht]
  \begin{minted}[linenos,frame=lines,framesep=2mm,fontsize=\footnotesize]{javascript}
extend nodes with {
  inbox : [timestamp, byte*]* = []
}
  \end{minted}
  \vspace{-5mm}
  \caption{Voorbeeld van een complex type}
  \label{lst:foo-complex_type}
\end{listing}
