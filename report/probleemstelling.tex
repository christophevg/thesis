%!TEX root=masterproef.tex

\chapter{Probleemstelling}
\label{chapter:probleemstelling}

Uit de bespreking van de context waarin deze thesis kadert, wordt duidelijk dat
inbraakdetectie bij draadloze sensornetwerken een dimensie complexer kan zijn
dan de overeenkomstige oplossingen in een klassiek netwerk. Ook de vergelijking
met de reeds bestudeerde mobiele ad-hoc netwerken houdt niet stand. De
bestaande technologie\"en en oplossingen kunnen blijkbaar moeilijk rechtstreeks
toegepast worden.

In dit hoofdstuk wil ik deze probleemstelling in detail bestuderen. Hierbij
vertrekken we in \ref{section:motes} van het lijdend voorwerp van deze thesis,
de sensorknoop. Wat maakt dit embedded systeem zo verschillend?

Sectie \ref{section:ids} gaat dieper in op de algemene problematiek van
inbraakdetectie en tracht uit de achtergrondinformatie de link te leggen naar
de sensorknopen.

Nog een stap hoger op de abstractieschaal komen we in sectie
\ref{section:algorithms} bij het beschrijven van de detectiealgoritmen. Ook op
dit niveau zullen we constateren dat belangrijke aspecten van het probleem
reeds ontstaan.

\section{Sensorknopen}
\label{section:motes}

\TODO

\section{Inbraakdetectie}
\label{section:ids}

\TODO

\section{Detectiealgoritmen}
\label{section:algorithms}

\TODO
