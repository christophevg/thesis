%!TEX root=masterproef.tex

\chapter{Probleemstelling}
\label{chapter:probleemstelling}

Uit de bespreking van de context waarin deze thesis kadert, wordt duidelijk dat
inbraakdetectie bij draadloze sensornetwerken een dimensie complexer kan zijn
dan de overeenkomstige oplossingen in een klassiek computer netwerk. In dit
hoofdstuk nemen we het volledige probleemgebied in beschouwing en duiden we de
essenti\"ele pijnpunten aan. Op basis van deze situatieschets zullen we in het
volgende hoofdstuk dan een oplossing voorstellen die beantwoordt aan deze
probleemstelling.

De wereld van DSN bestaat uit meer dan louter de sensorknopen waar we
logischerwijs direct aan denken. Toegegeven, ze zijn natuurlijk de elementaire
bouwsteen en vormen daarmee ook het eerst belangrijke niveau. In sectie
\ref{section:problem-hardware} vertrekken met we ons onderzoek bij de hardware,
ofwel de sensorknoop zelf.

Hardware zonder software is zoals een cafe zonder bier. \'E\'en niveau boven de
hardware vinden we de software die de sensorknoop in staat stelt om zijn taken
uit te voeren. In sectie \ref{section:problem-software} bekijken we de software
in het algemeen, waarbij zowel het besturingssysteem als de toepassing aan bod
komt.

Naast de infrastructuur hebben we natuurlijk nood aan de nodige software om aan
inbraakdetectie te doen. De eerste stap van elke ontwikkeling is een analyse
van het probleem en een beschrijving van de beoogde implementatie. In het geval
van inbraakdetectie in DSN, moeten we ons hier voorlopig nog beroepen op
onderzoek. Sectie \ref{section:problem-research} vertrekt daarom vanuit de
situatie van onderzoekers.

Met een sensorknoop voorzien van software onder de arm en een analyse en
beschrijving van algoritmes te beschikking, hebben we nog iemand nodig om de
nodige inbraakdetectie-software op een correcte manier te bouwen. Sectie
\ref{section:problem-develop} bekijkt het probleem door de ogen van de
ontwikkelaar.

Tot slot is de ontwikkelaar zelden de eigenaar of uitbater van het netwerk. Op
het hoogste niveau vinden we de persoon die uiteindelijk de reden is van het
bestaan van de sensorknopen, de software en het netwerk als een geheel. De
problemen op het niveau van de uitbating worden bekeken in sectie
\ref{section:problem-operations}.


\section{Sensorknopen}
\label{section:problem-hardware}

\TODO

- static <-> MANET
- wireless network/broadcasting <-> peer-peer (MANET)
- node capture

\section{Software}
\label{section:problem-software}

\TODO

\section{Onderzoek}
\label{section:problem-research}

\TODO

\section{Ontwikkeling}
\label{section:problem-develop}

\TODO

\section{Uitbating}
\label{section:problem-operations}

\TODO
